\section{发展历程简介}
目前的设想是从两个维度梳理公钥加密的发展历史: 功能性的丰富和安全性的增强. 

自Diffie和Hellman的划时代论文~\cite{DH-IEEE-IT-1976}后, 
公钥密码学的发展一日千里、日新月异, 热潮持续至今, 始终是现代密码学的核心和重要技术的摇篮. 

公钥密码的发展大体可以分为两条主线: 一条是安全性的增强, 从最初的直觉安全演进到严格健壮的语义安全, 
再到不可区分选择密文安全和各类超越传统安全模型的高级安全性, 如抗泄漏安全、抗篡改安全和消息依赖密钥安全; 
另一条是功能性的丰富, 从最初的一对一加解密到基于身份加密, 再到属性加密乃至极致泛化的函数加密. 

\subsection{安全性增强}
最初的RSA PKE只满足朴素单向安全. 
Goldwasser和Micali思考了公钥加密应该满足何种最低安全性这一问题, 提出了严格的安全模型, 设计了首个可证明安全的公钥加密方案Goldwasser-Micali PKE. 



\subsection{功能性丰富}
将加上与易红旭和王煜宇共同撰写的综述