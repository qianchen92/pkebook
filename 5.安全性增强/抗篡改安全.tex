\section{抗篡改安全}
%写作思路: 先参考Damg{\aa}rd等的~\cite{DFMV-ASIACRYPT-2013}简介PKE的抗篡改安全, 然后可以介绍一下Wee的~\cite{Wee-PKC-2012}, 然后再介绍你的PKC或者直接介绍我们的NMF.

相关密钥攻击 (Related-Key Attacks, RKAs) 最早由Biham~\cite{Biham-JOC-1994}和Knudsen~\cite{Knudsen-AUSCRYPT-1992}提出的. 2003年,Bellare和Kohno~\cite{BK-EUROCRYPT-2003}给出它的形式化定义. 早期设计的密码算法仅能抵抗简单的线性密钥篡改攻击, 例如Bellare和Cash~\cite{BC-CRYPTO-2010}提出的基于DDH假设的抗相关密钥攻击的伪随机函数 (RKA-PRFs). 2011年, Bellare等人~\cite{BCM-ASIACRYPT-2011}给出如何从RKA-PRFs和其它非RKA安全的密码算法来实现RKA安全的密码算法, 包括公钥加密、对称加密、签名和基于身份加密. 同年, Applebaum等人~\cite{AHI-ITCS-2011}提出基于LPN和LWE假设的抗线性密钥篡改语义安全对称加密方案. 2012年, Wee~\cite{Wee-EUROCRYPT-2012}提出利用特殊性质的自适应单向陷门函数构造抗线性篡改的RKA-CCA公钥加密方案, 并给出基于因子分解、DBDH和LWE等困难问题的具体实现. 

\subsection{安全模型}
根据文献~\cite{BCM-ASIACRYPT-2011}, 一个密码系统通常由系统参数、算法 (程序实现的代码)和密钥 (公钥/私钥)三部分组成. 公钥/私钥是最有可能受到RKA攻击的, 而系统参数和算法假定是不受攻击的. 这是因为, 系统参数并不包含用户的密钥信息, 与用户是独立的. 它可以在用户密钥选取之前确定并且可以嵌入到算法的实现代码中. 令$\Phi = \{\phi: \mathcal{SK} \rightarrow \mathcal{SK}\}$是一个从密钥空间$\mathcal{SK}$到自身的变换函数族.一个公钥加密方案$\text{PKE}$的RKA-CCA安全模型的定义如下:
\begin{trivlist}
\item \textbf{RKA-CCA安全性.} 定义公钥加密方案$\text{PKE}$的RKA-CCA敌手$\mathcal{A} = (\mathcal{A}_1, \mathcal{A}_2)$的优势函数如下: 
\begin{displaymath}
	\mathsf{Adv}_{\mathcal{A}, \Phi, \text{PKE}}^{\text{RKA-CCA}}(\kappa) = \Pr \left[ \beta' = \beta:~
	\begin{array}{ll}
		& pp \leftarrow \mathsf{Setup}(1^\kappa); \\		
		& (pk, sk) \leftarrow \mathsf{KeyGen}(pp);\\
		& (m_0, m_1, state) \leftarrow \mathcal{A}_1^{\blue{\mathcal{O}_{\text{rka}}}(\cdot)}(pp, pk), s.t. |m_0| = |m_1|; \\
		& \beta \sample \{0,1\};\\ 
        & c^* \leftarrow \mathsf{Encrypt}(pk, m_\beta);\\
        & \beta' \leftarrow \mathcal{A}_2^{\blue{\mathcal{O}_{\text{rka}}}(\cdot)}(pp, pk, state, c^*);
	\end{array} 
\right] - \frac{1}{2},
\end{displaymath}
上述定义中, 敌手$\mathcal{A}$在接收到挑战密文前后两阶段都可以访问密钥篡改谕言机同时获得在篡改密钥下的解密结果. 具体地, 谕言机$\mathcal{O}_{\text{rka}}(\cdot)$的输入为一对篡改函数和密文$(\phi, c)$, 其中$\phi \in \Phi$, 输出为$\mathsf{Decrypt}(\phi(sk), c)$. 在第2阶段询问中, 敌手不能进行满足条件$(\phi(sk), c) = (sk, c^*)$的询问, 否则敌手直接获取了挑战密文的解密结果, 使安全模型失去了实际意义. 如果对于任意的PPT敌手$\mathcal{A}$, 优势函数$\mathsf{Adv}_{\mathcal{A}, \Phi, \text{PKE}}^{\text{RKA-CCA}}(\kappa)$是可忽略的, 则称公钥加密方案$\text{PKE}$是$\Phi$-RKA-CCA安全的.
\end{trivlist}

\begin{note}
在RKA攻击中, $\Phi$称为密钥篡改函数族. 如果对于所有密钥$sk\in\mathcal{SK}$及所有不同的篡改函数$\phi,\phi' \in \Phi$都有$\phi(sk) \neq \phi'(sk)$, 则密钥篡改函数族$\Phi$称为``claw-free''的. 在已有的RKA安全加密或其他密码方案中, 大部分方案仅能抵抗这类篡改函数. "Claw-free"篡改函数是一种特殊的函数, 在实际中, 绝大部分篡改攻击函数都是非"claw-free"的. 从前面的定义可以看出RKA-CCA与IND-CCA安全性之间有着密切的联系. 在两种模型中, 敌手都可以访问解密服务. 不同之处在于RKA敌手还可以访问篡改密钥下的解密服务. 此外, 只要$\phi(sk) \neq sk$, 敌手是可以访问挑战密文的解密服务的. 这也是RKA-CCA安全性比IND-CCA安全性更难实现的原因之一. 如果密钥篡改函数仅包含单位函数$1_\phi$, 则$\{1_\phi\}$-RKA-CCA等价于IND-CCA.
\end{note}

\subsection{Wee12方案}
\subsubsection{自适应单向陷门关系的特殊性质}
自适应单向陷门关系(ATDR)在构造IND-CCA安全公钥加密方面具有强大的优势. 而RKA-CCA安全的PKE本身也是IND-CCA安全的, 那么一个自然的问题是自适应单向陷门关系能否用于构造RKA-CCA安全的公钥加密方案, 需要满足哪些特殊的性质, 篡改函数集的形式如何呢. 2012年, Wee~\cite{Wee-EUROCRYPT-2012}给出了这些问题的答案, 提出一种基于自适应单向陷门关系的RKA-CCA安全公钥加密方案的通用构造. 带标签自适应单向陷门关系在随机采样和求逆算法中会额外输入一个标签, 而该标签与自适应单向陷门关系的陷门无关. 一个带标签自适应单向陷门关系$\text{ATDR} = (\mathsf{Setup}, \mathsf{KeyGen}), \mathsf{Sample}, \mathsf{TdInv})$需要满足以下两个额外的性质:
\begin{itemize}
\item 密钥同态性 ($\Phi$-Key Homomorphism): 对于任意$\phi \in \Phi$和任意的陷门$td$, 标签$tag$, 关系值$y$, 存在一个PPT算法$T$, 使得
\[
\mathsf{TdInv}(\phi(td), tag, y) = \mathsf{TdInv}(td, tag, T(pp, \phi, tag, y)).
\]

\item 指纹识别性 ($\Phi$-Fingerprinting): 类似于指纹认证, 对于一个固定的关系值 (指纹) $y^*$, 对陷门的任何篡改, 通过求逆算法都可以被检测出来. 具体地, 定义敌手$\mathcal{A}$的优势函数如下:
\begin{displaymath}
	\mathsf{Adv}_{\mathcal{A}, \Phi, \text{ATDR}}^{\text{FP}}(\kappa) = \Pr \left[ 
\begin{array}{l}
\mathsf{TdInv}(\phi(td), tag^*, y^*) \neq \bot\\
\land \phi \in \Phi \land \phi(td) \neq td
\end{array}
:~
	\begin{array}{ll}
		& pp \leftarrow \mathsf{Setup}(1^\lambda); \\
		& tag^* \leftarrow \mathcal{A}(pp); \\
		& (ek, td) \leftarrow \mathsf{KeyGen}(pp);\\
		& (s^*, y^*) \sample \mathsf{Sample}(td, tag^*); \\
		& \phi \leftarrow \mathcal{A}(pp, ek, td, y^*);
	\end{array} 
\right]
\end{displaymath}
对于任意PPT敌手$\mathcal{A}$, 如果优势函数$\mathsf{Adv}_{\mathcal{A}, \Phi, \text{ATDR}}^{\text{FP}}(\kappa)$都是可忽略的, 则称$\text{ATDR}$是$\Phi$-Fingerprinting的.
\end{itemize}

\begin{note}
上面这两个额外的性质为RKA-CCA安全性的证明提供了一种简洁的解决办法. 首先,密钥同态性实际上提供了一种通过原始求逆谕言机$\mathsf{TdInv}(td, tag, \cdot)$来回答在篡改密钥$\phi(td)$下的求逆询问. 指纹识别性实际上保证了敌手不能查询挑战关系值在原始陷门下的求逆询问. 当用ATDR构造IND-CCA安全的公钥加密方案时, 这两种额外的性质可以直接用于RKA-CCA安全性中, 从而使得构造IND-CCA安全的公钥加密方案也是RKA-CCA安全的. 
\end{note}

\begin{note}
在$\Phi$-Fingerprinting性质中, 敌手是知道陷门$td$的. 这一条件在后面的证明中至关重要, 等价于挑战者知自适应单向陷门关系的陷门, 从而可以模拟解密谕言机.
\end{note}

\subsubsection{基于自适应单向陷门关系的RKA-CCA方案}
下面介绍Wee的通用构造方案. 该方案还需要一个强不可伪造一次签名方案 (Strong One-Time Signatures) $\text{OTS} = (\mathsf{Setup}, \mathsf{KeyGen}, \mathsf{Sign}, \mathsf{Verify})$. 其中, $\mathsf{Setup}(1^\kappa)$输出签名方案的系统参数$pp$; $\mathsf{KeyGen}(pp)$输出签名公钥$vk$和私钥$sigk$; $\mathsf{Sign}(sigk, M)$输出消息$M$的签名$\sigma$; $\mathsf{Verify}(vk, M, \sigma)$输出1当且仅当$\sigma$是$M$的一个合法签名. 一次签名方案的敌手$\mathcal{A}$的优势函数定义如下:
\begin{displaymath}
	\mathsf{Adv}_{\mathcal{A}, \text{OTS}}(\kappa) = \Pr \left[ 
\begin{array}{l}
\mathsf{Verify}(vk, M', \sigma') = 1\\
\land (M', \sigma') \neq (M, \sigma)
\end{array}
:~
	\begin{array}{ll}
		& pp \leftarrow \mathsf{Setup}(1^\lambda); \\
		& (vk, sigk) \leftarrow \mathsf{KeyGen}(pp);\\
		& M \leftarrow \mathcal{A}(vk);\\
		& \sigma \sample \mathsf{Sign}(sigk, M); \\
		& (M', \sigma') \leftarrow \mathcal{A}(\sigma);
	\end{array} 
\right]
\end{displaymath}
对于任意PPT敌手, 如果优势函数$\mathsf{Adv}_{\mathcal{A}, \text{OTS}}(\kappa)$是可忽略的, 则称一次签名方案是强不可伪造的.

\begin{construction}[Wee RKA-CCA PKE]\label{construction:Wee-PKE}
\begin{itemize}
\item $\mathsf{Setup}(1^\kappa)$: 运行$\text{ATDR}.\mathsf{Setup}(1^\kappa)$和$\text{OTS}.\mathsf{Setup}(1^\kappa)$, 分别输出ATDR的系统参数$pp_1$和$pp_2$. 选择一个伪随机函数$\mathsf{G}: X \rightarrow \{0, 1\}^m$, 其中$X$为ATDR的原像空间. 输出公钥加密方案的公开参数$pp = (pp_1, pp_2, \mathsf{G})$. 其中$\{0, 1\}^m$作为明文空间.

\item $\mathsf{KeyGen}(pp)$: 运行$(ek, td) \leftarrow \text{ATDR}.\mathsf{KeyGen}(pp)$, 输出公钥$pk := ek$和私钥$sk: = td$. 
  
\item $\mathsf{Encrypt}(pk, M)$: 以公钥$pk := ek$和明文$M \in \{0, 1\}^m$为输入, 执行如下步骤:
\begin{enumerate}
    \item 运行$(vk, sigk) \leftarrow \text{OTS}.\mathsf{KeyGen}(pp_2)$生成一次签名的公钥和私钥;
    \item 运行$(x, y) \leftarrow \text{ATDR}.\mathsf{Sample}(ek, vk)$生成一个随机采样$(x, y)$; 
    \item 计算$\psi = \mathsf{G}(x)\oplus M$;
    \item 运行$\sigma \leftarrow \text{OTS}.\mathsf{Sign}(sigk, y||\psi)$;
    \item 输出密文$C = (vk, \sigma, y, \psi)$. 
\end{enumerate} 

\item $\mathsf{Decrypt}(sk, C)$: 以私钥$sk := td$和密文$C = (vk, \sigma, y, \psi)$为输入, 执行如下步骤:
\begin{enumerate}
    \item 验证$\text{OTS}.\mathsf{Verify}(vk, y||\psi, \sigma) = 1$. 若不成立, 则返回$\bot$, 否则执行后续步骤;
    \item 计算$x \leftarrow \text{ATDR}.\mathsf{TdInv}(td, vk, y)$. 若$x = \bot$, 则返回$\bot$, 否则执行后续步骤; 
    \item 计算$M' = \mathsf{G}(x) \oplus \psi$并返回明文$M'$. 
\end{enumerate} 
\end{itemize}
\end{construction}

\begin{trivlist}
\item \textbf{正确性.} 上述加密方案的正确性可由自适应单向陷门关系的正确性直接推导出来. 下面主要介绍方案的RKA-CCA安全性的证明.
\end{trivlist}

\begin{theorem}\label{theorem:Wee-PKE}
如果$\text{ATDR}$是一个自适应单向陷门关系族, 且满足$\Phi$-密钥同态性和$\Phi$-指纹识别性, $\text{OTS}$是一个强不可伪造一次签名方案, 那么定义~\ref{definition:Wee-PKE}中的公钥加密方案是$\Phi$-RKA-CCA
安全的. 特别地, 对于任意攻击方案$\Phi$-RKA-CCA安全性的敌手$\mathcal{A}$, 若$\mathcal{A}$询问解密谕言机$\mathcal{O}_{\text{rka}}(\cdot)$的次数最多为$Q$次, 则敌手成功的优势满足如下关系式:
\[
\mathsf{Adv}_{\mathcal{A}, \Phi, \text{PKE}}^{\text{RKA-CCA}}(\kappa) \leq  \mathsf{Adv}_{\mathcal{B}_1, \text{OTS}}(\kappa) + \mathsf{Adv}_{\mathcal{B}_2, \Phi, \text{ATDR}}^{\text{FP}}(\kappa) + \mathsf{Adv}_{\mathcal{B}_3, \text{ATDR}}^{\text{OW}}(\kappa), 
\]
其中, $\mathcal{B}_1$是攻击一次签名强不可伪造性的敌手, $\mathcal{B}_2$是攻击自适应单向陷门关系$\Phi$-指纹识别性的敌手,$\mathcal{B}_3$是攻击自适应单向陷门关系单向性的敌手.  $\mathcal{B}_1$、$\mathcal{B}_2$和$\mathcal{B}_3$的运行时间与敌手$\mathcal{A}$的时间接近或增加与$Q$线性增长的多项式时间开销.
\end{theorem}

\begin{proof}
令$S_i$表示敌手在$\text{Game}_i$中的成功事件. 以游戏序列的方式组织证明如下: 

\begin{trivlist}
\item $\text{Game}_0$: 该游戏是标准的RKA-CCA游戏, 挑战者$\mathcal{CH}$和敌手$\mathcal{A}$交互如下: 
\begin{itemize}
\item 初始化: $\mathcal{CH}$运行$\mathsf{Setup}(1^\kappa)$生成公开参数$pp$, 
		同时运行$\mathsf{KeyGen}(pp)$生成公私钥对$(pk, sk)$. $\mathcal{CH}$将$(pp, pk)$发送给$\mathcal{A}$. 

\item 询问: 对于敌手的任意询问$(\phi, C)$, $\mathcal{CH}$首先判断$\phi \in \Phi$是否成立, 若不成立则返回$\bot$, 否则返回$\mathsf{Decrypt}(\phi(sk), C)$的解密结果.  

\item 挑战: $\mathcal{A}$选择$M_0, M_1 \in \mathbb{G}$并发送给$\mathcal{CH}$. $\mathcal{CH}$选择随机比特$\beta \in \{0,1\}$, 作如下计算:
\begin{enumerate}
    \item 运行$(vk^*, sigk^*) \leftarrow \text{OTS}.\mathsf{KeyGen}(pp_2)$生成一次签名的公钥和私钥;
    \item 运行$(x^*, y^*) \leftarrow \text{ATDR}.\mathsf{Sample}(ek, vk^*)$生成一个随机采样$(x^*, y^*)$; 
    \item 计算$\psi^* = \mathsf{G}(x^*)\oplus M_\beta$;
    \item 运行$\sigma^* \leftarrow \text{OTS}.\mathsf{Sign}(sigk^*, y^*||\psi^*)$;
    \item 输出密文$C^* = (vk^*, \sigma^*, y^*, \psi^*)$并发送给$\mathcal{A}$. 
\end{enumerate} 

\item 猜测: $\mathcal{A}$输出对$\beta$的猜测$\beta'$. $\mathcal{A}$成功当且仅当$\beta' = \beta$. 
\end{itemize} 
根据定义, 我们有: 
\begin{equation*}
	\mathsf{Adv}_{\mathcal{A}, \Phi, \text{PKE}}^{\text{RKA-CCA}}(\kappa) = |\Pr[S_0] - 1/2|
\end{equation*}

\item $\text{Game}_1$: 该游戏与$\text{Game}_0$的唯一不同在于拒绝解密查询的条件.	对于敌手提交的解密查询$(\phi, C)$, 其中$C = (vk, \sigma, y, \psi)$, 若$vk = vk^*$, 则$\mathcal{CH}$直接拒绝提供解密服务并返回$\bot$. 若$vk \neq vk^*$, 则$\mathcal{CH}$提供的解密查询服务与$\text{Game}_0$完全一样. 下面分四种情况讨论敌手在两个连续游戏中的视图之间的区别.
\begin{itemize}
\item Case~1: $vk \neq vk^*$. 在这种情况下, 游戏$\text{Game}_0$与$\text{Game}_1$中的解密服务是完全一样的.

\item Case~2: $vk = vk^*$且$(\sigma, y||\psi) = (\sigma^*, y^*||\psi^*)$且$\phi(sk) = sk$. 该情况实际上等价于$(\phi(sk), C) = (sk, C^*)$. 根据RKA-CCA安全模型的定义, 这种情况在两个游戏中都是不允许进行解密查询的.

\item Case~3: $vk = vk^*$且$(\sigma, y||\psi) \neq (\sigma^*, y^*||\psi^*)$. 根据一次签名的强不可伪造性, 我们可以直接证明$(y||\psi, \sigma)$通过验证的概率是可忽略的. 具体地, 我们有
\[
\Pr \left[\mathsf{Verify}(vk, y||\psi, \sigma) = 1 \right] \leq \mathsf{Adv}_{\mathcal{B}_1, \text{OTS}}(\kappa).
\]

\item Case~4: $vk = vk^*$且$(\sigma, y||\psi) = (\sigma^*, y^*||\psi^*)$且$\phi(sk) \neq sk$. 根据$\text{ATDR}$的$\Phi$-指纹识别性, 解密服务在计算$x \leftarrow \text{ATDR}.\mathsf{TdInv}(td, vk, y)$时, $x \neq \bot$的概率是可忽略的. 若$x = \bot$, 则解密谕言机直接返回$\bot$. 此时, 两个游戏中解密谕言机返回的结果是一样的. 特别地, 我们有
\[
\Pr \left[\text{ATDR}.\mathsf{TdInv}(\phi(sk), vk^*, y) \neq \bot \right] \leq \mathsf{Adv}_{\mathcal{B}_2, \Phi, \text{ATDR}}^{\text{FP}}(\kappa).
\]
\end{itemize}
由上述分析可知, 敌手在两个游戏中的视图区别是可忽略的. 具体地, 我们有
\begin{equation*}
|\Pr[S_0] - \Pr[S_1]| \leq \mathsf{Adv}_{\mathcal{B}_1, \text{OTS}}(\kappa) + \mathsf{Adv}_{\mathcal{B}_2, \Phi, \text{ATDR}}^{\text{FP}}(\kappa).
\end{equation*}

\item $\text{Game}_2$: 该游戏与$\text{Game}_1$的唯一不同在于挑战者利用$\text{ATDR}$的自适应性来回答解密询问.对于敌手提交的解密查询$(\phi, C)$, 其中$C = (vk, \sigma, y, \psi)$, 解密谕言机的工作方式如下:
\begin{itemize}
\item 若$vk = vk^*$或者$\text{OTS}.\mathsf{Verify}(vk, y||\psi, \sigma) = 0$, 返回$\bot$; 

\item 计算$x :=\text{ATDR}.\mathsf{TdInv}(td, vk, T(pp, \phi, vk, y))$. 如果$x = \bot$, 返回$\bot$;

\item 否则, 计算并返回$M' = \mathsf{G}(x) \oplus \psi$.
\end{itemize}
根据$\Phi$-密钥同态性, 我们有$\text{ATDR}.\mathsf{TdInv}(td, vk, T(pp, \phi, vk, y)) = \text{ATDR}.\mathsf{TdInv}(\phi{td}, vk, y)$. 也就是说, 挑战者在$\text{Game}_2$中模拟的解密谕言机和$\text{Game}_1$中的是完全一致的. 因此, 我们有
\[
\Pr[S_1] = \Pr[S_2].
\]

\item $\text{Game}_3$: 该游戏与$\text{Game}_2$的唯一不同在于挑战密文中$\psi^*$的计算方式. $\mathcal{CH}$随机选择$K \sample \{0, 1\}^m$, 计算$\psi^* = K \oplus M_\beta$. 根据$\text{ATDR}$的自适应单向性及函数$\mathsf{G}$输出结果的伪随机性, 我们可以证明敌手在这两个连续游戏中的视图区别不超过$\mathsf{Adv}_{\mathcal{B}_3, \text{ATDR}}^{\text{OW}}(\kappa)$. 故, 我们有
\begin{equation*}
|\Pr[S_2] - \Pr[S_3]| \leq \mathsf{Adv}_{\mathcal{B}_3, \text{ATDR}}^{\text{OW}}(\kappa),
\end{equation*}
其中$\mathcal{B}_3$是攻击自适应单向陷门关系的单向性的PPT敌手.
\end{trivlist}

在最后一个游戏中, 由于$\psi^*$的分布与挑战比特$\beta$完全独立不相关, 从而敌手在游戏中的优势为0, 即
\[
Pr[S_3] = \frac{1}{2}.
\]

综上, 定理~\ref{theorem:Wee-PKE}得证. \qed
\end{proof}

\subsection{CQZ+22方案}
上世纪90年代, 哥德尔奖得主Cynthia Dwork和Moni Naor引入的不可延展性是密码学中单向性和伪随机性之外的另一重要性质, 该性质精准刻画了密码组件输入/输出之间的独立性. 已有的研究工作考察了加密、承诺、零知识证明、编码、程序混淆的不可延展性, 然而一直未涉及密码学乃至计算机科学中最基本的函数. 函数的不可延展性与单向性之间存在怎样的关联以及如何构造高效的不可延展函数均是未解的公开问题.

2022年, Chen等人~\cite{CQZDC-JOC-2022}在PKC 2016工作~\cite{Chen-PKC-2016}的基础上进一步完善了不可延展函数的性质及构造, 成功解决了上述问题. 在理论层面, 首次绘制出函数不可延展性与单向性之间的清晰图景, 通过巧妙结合方程求解技巧和变换集代数性质, 建立起不可延展函数与单向函数之间的关联, 并分别在标准模型和随机谕言机模型中给出了通用构造, 解决了美国Georgia Tech 大学 Alexandra Boldyreva 教授和德国 Bochum 大学 Eike Kiltz 教授等著名密码学家提出的公开问题. 在应用层面, 不仅直接蕴含了密码谜题的高效设计, 还深度揭示了不可延展函数在抗篡改安全中的强力应用: (1)证明了对于代数诱导的变换集, 抗非平凡拷贝攻击属于密码方案的内蕴性质, 从而直接提升了一大批密码方案的抗相关密钥攻击安全性; (2)构造出了迄今为止效率和安全均最优的认证密钥导出函数, 提供了将传统安全提升为抗篡改安全的关键技术工具.

\subsubsection{不可延展函数}
\begin{definition}[有效计算函数]
一个有效计算函数族$\mathcal{F}$包含以下三个概率多项式时间算法$(\mathsf{Gen}, \mathsf{Eval}, \mathsf{Vefy})$:  
\begin{itemize}
\item $\mathsf{Gen}(1^\kappa)$: 输入安全参数$\kappa$, 输出一个函数索引$s$. 每个函数索引$s$定义了一个函数$f_s: X_s \rightarrow Y_s$. 该函数可以是确定性的也可以是随机性的.

\item $\mathsf{Eval}(s, x)$: 输入函数索引$s$和一个元素$x \in X_s$, 输出函数的一个像值$y \leftarrow f_s(x)$. 令$\text{supp}(f_s(x))$是随机变量$f_s(x)$的支撑集. 如果$f_s$是确定的, 则 $\text{supp}(f_s(x))$缩减为包含唯一像值$f_s(x)$的集合.

\item $\mathsf{Vefy}(s, x, y)$: 输入函数索引$s$和$(x, y) \in X_s \times Y_s$, 当$y \in \text{supp}(f_s(x))$时, 输出``1'', 否则, 输出``0''. 对于确定性函数, 可以直接通过重新计算$x$的像值来验证.  
\end{itemize}
\end{definition}

\begin{note}
上述定义统一了确定函数和随机函数的概念. 对于任意两个不同的原像$x_1, x_2 \in X_s$, 如果$\text{supp}(f(x_1))$和$\text{supp}(f(x_2))$没有交集, 则$f$是单射函数.  
\end{note}

下面介绍有效计算函数的单向性(One-Wayness)和不可延展性(Non-malleability).
\begin{definition}[单向性和自适应单向性]\label{def:OW}
定义$\mathcal{F}$的单向性敌手的优势函数如下:
\begin{displaymath}
	\mathsf{Adv}_{\mathcal{A}, \mathcal{F}}^{\text{ow}}(\kappa) =
		\Pr \left[x \in  f_s^{-1}(y^*):
		\begin{array}{ll}
           	& s \leftarrow \mathsf{Gen}(1^\kappa);\\
			& x^* \sample X_s, y^* \leftarrow f_s(x^*);\\
			& x \leftarrow \mathcal{A}(f_s, y^*);
		\end{array} 
		\right].
\end{displaymath}
对于任意PPT敌手$\mathcal{A}$, 如果优势函数$\mathsf{Adv}_{\mathcal{A}, \mathcal{F}}^{\text{ow}}(\kappa)$关于安全参数$\kappa$是可忽略的, 则$\mathcal{F}$是单向的. 如果允许$\mathcal{A}$访问除了$y^*$之外的任意$y$的求逆谕言机$\mathcal{O}_\mathsf{inv}$情况下, $\mathcal{F}$仍然是单向的, 则称$\mathcal{F}$是自适应单向的.    
\end{definition}

\begin{definition}[不可延展性和自适应不可延展性]\label{def:NM}
令$\Phi$是一个定义域$X$上的变换函数集. 定义$\mathcal{F}$的不可延展性敌手的优势函数如下:
\begin{displaymath}
	\mathsf{Adv}_{\mathcal{A}, \mathcal{F}}^{\text{nm}}(\kappa) =
		\Pr \left[
		\begin{array}{c}
			\phi \in \Phi \wedge \mathsf{Vefy}(s, \phi(x^*), y)=1 \\
			\wedge (\phi, y) \neq (\mathsf{id}, y^*)
		\end{array} :
		\begin{array}{ll}
         		& s \leftarrow \mathsf{Gen}(1^\kappa);\\
			& x^* \sample X_s, y^* \leftarrow f_s(x^*);\\
			& (\phi, y) \leftarrow \mathcal{A}(f_s, y^*);
		\end{array} 
		\right].
\end{displaymath}
对于任意PPT敌手$\mathcal{A}$, 如果优势函数$\mathsf{Adv}_{\mathcal{A}, \mathcal{F}}^{\text{nm}}(\kappa)$关于安全参数$\kappa$是可忽略的, 则$\mathcal{F}$是$\Phi$-不可延展的. 如果允许$\mathcal{A}$访问除了$y^*$之外的任意$y$的求逆谕言机$\mathcal{O}_\mathsf{inv}$情况下, $\mathcal{F}$仍然是不可延展的, 则称$\mathcal{F}$是自适应$\Phi$-不可延展的.     
\end{definition}

\begin{note}
一般来说, 一个函数族的定义域和值域范围依赖函数索引. 为简便起见, 我们可以假设对于所有函数索引$s$, 定义域和值域都是不变的. 从而将$X_s$, $Y_s$和$f_s$分别简写为$X$, $Y$和$f$. 在不可延展函数中, 存在一些不可能转换函数类, 如单位变换$\mathsf{id}$和常量变换$\phi_c$, 无法实现不可延展性. 敌手可以输出$(\mathsf{id}, y^*)$或$(\phi_c, f(c))$从而赢得游戏. 因此, 给出一个可能实现不可延展性的变换函数集$\Pi$是非常重要的.
\end{note}

\begin{definition}[一般变换函数集]
定义满足下面两个性质的变换函数集$\Phi_\textup{brs}^\textup{srs}$: 
\begin{itemize}
\item 有界根集合(Bounded Root Space): 令$r(\kappa)$是安全参数$\kappa$的一个变量, $R_\phi = \{x \in X : \phi(x) = 0\}$. 如果$|R_\phi| \leq r(\kappa)$, 则$\phi$最多有$r(\kappa)$个根. 如果对于每个$\phi \in \Phi$和$\phi_c \in \mathsf{cf}$, 变换函数$\phi' = \phi - \phi_c$和$\phi'' = \phi - \mathsf{id}$都最多有$r(\kappa)$个根, 那么称变换函数集$\Phi$有$r(\kappa)$-有界根集.

\item 可采样根集合(Sampleable Root Space): 如果存在一个PPT算法$\mathsf{SampRS}$, 输入$\phi$, 均匀随机地输出集合$R_\phi$中的一个元素, 则称变换函数$\phi$有一个可采样根集. 如果对于每个$\phi \in \Phi$和$\phi_c \in \mathsf{cf}$, 复合函数$\phi' = \phi - \phi_c$和$\phi'' = \phi - \mathsf{id}$都有可采样根集, 那么称变换函数集$\Phi$有可采样根集.
\end{itemize}
\end{definition}

\begin{note}
变换函数集$\Phi_\text{brs}^\text{srs}$非常强大, 几乎包含了所有除去单位变换$\mathsf{id}$和常量变换$\mathsf{cf}$的代数诱导变换函数集, 如线性变换集$\Phi^\text{lin} = \{\phi_a : \phi_a(x) = a + x\}_{a \in \mathbb{G}}$, 仿射变换集$\Phi^\text{aff} = \{\phi_{a,b} : \phi_{a,b}(x) = ax + b\}_{a,b \in \mathbb{R}}$和多项式变换集$\Phi^{\text{poly}(d)} = \{\phi_q : \phi_q(x) = q(x)\}_{q \in \mathbb{F}_d(x)}$, 其中$\mathbb{G}$是一个群, $\mathbb{R}$是一个环, $\mathbb{F}_d(x)$是有限域$\mathbb{F}$上次数不超过$d$的多项式集. 
\end{note}

\subsubsection{不可延展性与单向性之间的关系}
首先, 概括地给出函数的不可延展性与单向性之间的关系, 如图~\ref{fig:ch5-relations}所示. 然后, 介绍图中转换关系相关几个重要引理.
\begin{figure}[!hbth]
\begin{center}
\begin{tikzpicture}
    \node [textnode, name=NM] {NM};
    \node [textnode, name=ANM, right of=NM, xshift=10em] {Adaptive NM}; 
    \node [textnode, name=OW, below of=NM, yshift=-8em] {OW};
    \node [textnode, name=AOW, right of=OW, xshift=10em] {Adaptive OW}; 

    \draw ($(NM.east)+(0em, 0.3em)$) edge[->, thick] node {$\slash$} node[yshift=2em] {poly-to-one\\$\forall \Phi_{\text{eff}} \supset \{\mathsf{id}\}$} ($(ANM.west)+(0em, 0.3em)$);
    \draw ($(ANM.west)+(0em,-0.3em)$) edge[->, thick] ($(NM.east)+(0em,-0.3em)$);
    \draw ($(NM.south)+(-0.3em, 0em)$) edge[->, thick] node [xshift=-3em]{poly-to-one\\$\forall \Phi_{\text{eff}} \supset \{\mathsf{id}\}$} ($(OW.north)+(-0.3em, 0em)$);
    \draw ($(OW.north)+(0.3em, 0em)$) edge[->, thick] node {$\slash$} node[xshift=1.5em] {$\Phi^\textup{xor}$} ($(NM.south)+(0.3em, 0em)$);

    \draw ($(OW.east)+(0em, 0.3em)$) edge[->, thick] node {$\slash$} ($(AOW.west)+(0em, 0.3em)$);
    \draw ($(AOW.west)+(0em,-0.3em)$) edge[->, thick] ($(OW.east)+(0em,-0.3em)$);
    \draw ($(ANM.south)+(-0.3em, 0em)$) edge[->, thick] node [xshift=-3em]{poly-to-one\\$\forall \Phi_{\text{eff}} \supset \{\mathsf{id}\}$} 
    	($(AOW.north)+(-0.3em, 0em)$);
    \draw ($(AOW.north)+(0.3em, 0em)$) edge[->, thick] node[xshift=2.5em] {injective\\$\Phi_\textup{brs}^\textup{srs} \cup \{\mathsf{id}\}$} ($(ANM.south)+(0.3em, 0em)$);


    \node [textnode, name=hintNMOW, right of=ANM, xshift=12em] {hinting OW/NM};
    \node [textnode, name=NMOW, below of=hintNMOW, yshift=-8em] {OW/NM}; 
    \draw ($(hintNMOW.south)+(-0.3em, -0em)$) edge[->, thick] node [xshift=-2em] {trivial} ($(NMOW.north)+(-0.3em, 0em)$);
	\draw ($(NMOW.north)+(0.3em, -0em)$) edge[->, thick] node [xshift=4em] {simulatable $\mathsf{hint}$} ($(hintNMOW.south)+(0.3em, 0em)$);
\end{tikzpicture}
\end{center}
\caption{不可延展函数与单向函数之间的关系}\label{fig:ch5-relations}
\end{figure} 

\begin{lemma}\label{lemma:NM-OW}
令$\mathcal{F}$是一个多对一函数族. 对于任意可实现的变换集$\Phi$, 则: $\Phi$-不可延展性$\Rightarrow$单向性.
\end{lemma}

\begin{proof}
假设$\mathcal{A}$是一个以不可忽略概率攻破$\mathcal{F}$单向性的敌手, 我们可以构造另一个算法$\mathcal{B}$以不可忽略的概率攻破$\mathcal{F}$的不可延展性. 算法$\mathcal{B}$模拟敌手$\mathcal{A}$在单向性游戏中的挑战者的过程如下: 
\begin{enumerate}
\item 初始化及挑战: 给定一个函数$f \leftarrow \mathcal{F}.\mathsf{Gen}(1^\kappa)$和一个像值$y^* \leftarrow f(x^*)$, 其中$x^* \sample X$, $\mathcal{B}$将$(f, y^*)$发送给敌手$\mathcal{A}$. 

\item 攻击: 当敌手$\mathcal{A}$输出一个解$x$时, $\mathcal{B}$随机选择一个变换函数$\phi \in \Phi \backslash \mathsf{id}$, 返回$(\phi, f(\phi(x))$作为对$\mathcal{F}$不可延展性的攻击结果.   
\end{enumerate}

由于$\mathcal{F}$是多对一的, 在$\mathcal{A}$攻击成功的条件下, 即$x \in f^{-1}(y^*)$, 我们有$\Pr[x = x^* |y^*] \geq 1/\mathsf{poly}(\kappa)$, 其中$x^* \sample X$. 这是因为最多有多项式个数$x$满足$f(x) = y^*$, 
并且每个$x$在敌手$\mathcal{A}$的角度看都是一样的. 因此, 如果$\mathcal{A}$能够以不可忽略的概率攻破$\mathcal{F}$的单向性, 那么算法$\mathcal{B}$同样能够以不可忽略的概率攻破$\mathcal{F}$的不可延展性. \qed    
\end{proof}

引理~\ref{lemma:NM-OW}反向结论并不成立, 也就是说一个函数满足单向性但并不一定满足不可延展性.
\begin{lemma}\label{lemma:OW-NM}
单向性$\nRightarrow$ $\Phi^{\text{xor}}$-不可延展性. 
\end{lemma}
\begin{proof}
\begin{figure}[!hbth]
\begin{center}
\begin{tikzpicture}
\node[textnode, name = x] {$x$}; 
\node[textnode, name = y, below of = x, yshift=-4em] {$f(x)$};
\draw (x) edge[->] node[left, name = f] {$f$} (y);  

\node[textnode, name = x_prime, right of = x, xshift=8em] {$x||b$}; 
\node[textnode, name = y_prime, below of = x_prime, yshift=-4em] {$f(x)||b$};
\draw (x_prime) edge[->] node[left, name = f_prime] {$f'$} (y_prime);  

\draw ($(f)+(1em, 0em)$) edge[->, densely dashed] ($(f)+(7em, 0em)$); 
\end{tikzpicture}
\end{center}
\caption{可延展函数示例}\label{fig:ch5-OW-NM}
\end{figure} 

引理~\ref{lemma:OW-NM}可通过反证法证明. 我们可以通过一个单向函数$f$构造另一个单向函数$f'$, 如图~\ref{fig:ch5-OW-NM}所示. 显而易见, $f'$仍然满足单向性, 但是不满足异或变换集下的不可延展性. \qed
\end{proof}

\begin{note}
思考一下, 不可延展性的反例还有哪些? 其实$\Phi$-同态的单向函数就一个很好的反例. 所谓$\Phi$同态性, 对于任意$\phi \in \Phi$和任意$x \in X$, 则有$f(\phi(x)) = \phi(f(x))$. 显而易见, $f$是单向的, 但不是$\Phi$-不可延展的.
\end{note}

\begin{lemma}\label{lemma:ANM-AOW}
对于$\mathcal{F}$上任意可实现的变换集$\Phi$ of $\mathcal{F}$, 自适应$\Phi$-不可延展性$\Rightarrow$ 自适应单向性.
\end{lemma}

\begin{proof}
引理~\ref{lemma:ANM-AOW}的证明可以直接通过引理~\ref{lemma:NM-OW}的结论推出. \qed
\end{proof}

\begin{lemma}\label{lemma:AOW-ANM}
当$\mathcal{F}$是单射函数时, 对于$\Phi = \Phi_\text{brs}^\text{srs} \cup \mathsf{id}$, $(q+1)$-自适应单向性$\Rightarrow$ $q$-自适应$\Phi$-不可延展性.  
\end{lemma}
\begin{proof}
假设$\mathcal{A}$是一个攻击$\mathcal{F}$自适应不可延展性的敌手. 我们可以构建一个攻击$\mathcal{F}$自适应单向性的敌手$\mathcal{B}$. $\mathcal{B}$按以下方式模拟$\mathcal{A}$在自适应不可延展游戏中的挑战者: 
\begin{trivlist}
\item 初始化及挑战: 给定$f \leftarrow \mathcal{F}.\mathsf{Gen}(1^\kappa)$和一个像值$y^* \leftarrow f(x^*)$, 其中$x^* \sample X$, $\mathcal{B}$将$(f, y^*)$作为挑战信息发送给敌手$\mathcal{A}$. 

\item 攻击: 当$\mathcal{A}$询问求逆谕言机时, $\mathcal{B}$将询问直接发送给自己的挑战者并将返回的结果发送给$\mathcal{A}$. 当$\mathcal{A}$输出一个攻击结果$(\phi, y) \neq (\mathsf{id}, y^*)$时, $\mathcal{B}$ 按下面的方式进行处理: 
	\begin{itemize}
	\item Case~1: $\phi = \mathsf{id} \wedge y \neq y^*$. $\mathcal{B}$询问求逆谕言机$\Oinv$, 获取$y$的逆$x$, 再将$x$输出作为$\mathcal{B}$的求解结果. 

	\item Case~2: $\phi \in \Phi_\text{brs}^\text{srs} \wedge y \neq y^*$. $\mathcal{B}$询问求逆谕言机$\Oinv$, 获得$y$的逆$x$, 再运行$\mathsf{SampRS}(\phi')$输出$\phi'(\alpha) = 0$的一个随机解, 其中$\phi'(\alpha) = \phi(\alpha) - x$. 

	\item Case~3: $\phi \in \Phi_\text{brs}^\text{srs} \wedge y = y^*$. $\mathcal{B}$运行$\mathsf{SampRS}(\phi'')$输出$\phi''(\alpha) = 0$的一个随机解, 其中$\phi''(\alpha) = \phi(\alpha) - \alpha$.  
	\end{itemize}
\end{trivlist}
下面分析$\mathcal{B}$的策略的正确性. 在$\mathcal{A}$攻击成功的条件下, 我们有$\mathsf{Vefy}(f, \phi(x^*), y) = 1$. 利用$\mathcal{F}$的单射性质, 对于情形1, 我们有$\mathsf{id}(x^*) = x^* = x$;   
对于情形2, 我们有$\phi(x^*) = x$, 即, $x^*$是$\phi'(\alpha) = 0$的一个解; 对于情形3, 我们有$\phi(x^*) = x^*$, 即, $x^*$是$\phi''(\alpha) = 0$的一个解. 将这三种情形结合起来, 利用变换集$\Phi_\text{brs}^\text{srs}$的BRS\&SRS性质, $\mathcal{B}$通过最多$(q+1)$次求逆询问输出正确解$x^*$的概率为$1/\mathsf{poly}(\kappa)$. 因此,如果$\mathcal{A}$攻破$q$-自适应不可延展性的概率是不可忽略的, 那么$\mathcal{B}$攻破$(q+1)$-自适应单向性的概率也是不可忽略的. 引理~\ref{lemma:AOW-ANM}得证. \qed 
\end{proof}

\begin{lemma}\label{lemma:NM-ANM}
当$\mathcal{F}$是多对一函数族时, 对于任意可实现的变换集$\Phi \supset \{\mathsf{id}\}$, $\Phi$-不可延展性$\nRightarrow$ 自适应$\Phi$-不可延展性.
\end{lemma}
\begin{proof}
我们通过寻找反例来证明引理~\ref{lemma:NM-ANM}的结论. 令$\mathcal{F} = (\mathsf{Gen}, \mathsf{Eval}, \mathsf{Vefy}, \mathsf{TdInv})$是一个带陷门的$\Phi$-不可延展函数族. 如图~\ref{fig:ch5-ANM}所示, 我们从$\mathcal{F}$构造另一个函数族$\mathcal{F}'$, 使得$\mathcal{F}'$仍然是$\Phi$-不可延展的, 但不是自适应$\Phi$-不可延展的.
\begin{figure}[!hbth] 
\begin{center}
\begin{tikzpicture}
\node[textnode, name = x1] {$x$}; 
\node[textnode, name = y1, below of = x1, yshift=-4em] {$f(x)$};
\draw (x1) edge[->] node[left, name = f] {$f$} (y1);  

\node[textnode, name = x2, right of = x1, xshift=8em] {$x$}; 
\node[textnode, name = y2, below of = x2, yshift=-4em] {$0||f(x)$};
\draw (x2) edge[->] node[left, name = f_prime] {$f'$} (y2);  

\draw ($(f)+(1em, 0em)$) edge[->, densely dashed] ($(f)+(7em, 0em)$); 
\end{tikzpicture}
\end{center}
\caption{自适应可延展函数示例}\label{fig:ch5-ANM}
\end{figure} 
对于任意$f \in \mathcal{F}$, 假设$f: \{0, 1\}^n \rightarrow \{0, 1\}^m$, 则$f' \in \mathcal{F}'$的定义如下:
\[
\begin{array}{ccccc}
f': & & \{0, 1\}^n &\rightarrow& \{0, 1\}^{m+1} \\
    & & x          &\rightarrow& 0 || f(x)
\end{array}
\]  
对于任意$y' = b||y \in \{0, 1\}^{m+1}$, $\mathcal{F}'$求逆函数的定义如下:
\[
f'^{-1}(y') = \left\{
\begin{array}{cc}
f^{-1}(y) & \text{if $b = 0$}\\
td      & \text{if $b = 1$}
\end{array}
\right.
\]
上面实际上定义了一个``危险的''求逆算法, 对于非法原像$b||y$(其中$b=1$), 求逆算法将直接输出$f$的陷门. 因此, 在自适应不可延展游戏中, 敌手可以通过这类``危险的''询问获取求逆陷门从而攻破$f'$的不可延展性. 而在不可延展性游戏里, 由于没有提供求逆谕言机, $f'$依然保持有不可延展性. 综上, 引理~\ref{lemma:NM-ANM}得证. \qed 
\end{proof}

\begin{lemma}[Computationally Simulatable Case]\label{lemma:ANM-computationally-simulatable}
令$\mathcal{F}$是一个单射函数族, $\mathcal{H}: X \rightarrow K$是$\mathcal{F}$的hardcore functions. 那么, 对于任意变换集$\Phi \subseteq \Phi_\text{brs}^\text{srs} \cup \mathsf{id}$, 自适应$\Phi$-不可延展性蕴含$\mathcal{H}$-hinting自适应$\Phi$-不可延展性 
\end{lemma}

\subsubsection{不可延展函数的构造}
下面利用自适应单向陷门函数和全除一有损陷门函数分别构造确定性不可延展函数和随机化不可延展函数. 对于确定性不可延展函数, 可以通过自适应单向陷门函数直接构造, 有如下结论:

\begin{theorem} 
令$\mathcal{F}$是一个单射的自适应单向陷门函数族, $\mathcal{H}$是与之相关的hardcore functions. 那么$\mathcal{F}$是一个自适应$\mathcal{H}$-hinting $\Phi$-不可延展的, 其中$\Phi =  \Phi_\text{brs}^\text{srs} \cup \{\mathsf{id}\}$.  
\end{theorem}

\begin{proof}
根据引理~\ref{lemma:AOW-ANM}, $\mathcal{F}$是自适应 $\Phi$-不可延展的. 根据引理~\ref{lemma:ANM-computationally-simulatable}, $\mathcal{F}$也是自适应$\mathcal{H}$-hinting 不可延展的. 定理得证! \qed  
\end{proof} 

随机化的不可延展函数的构造需要用到两个密码工具: 全除一有损函数~\cite{Qin-PKC-2014}(可以看作是不带陷门的全除一有损单向陷门函数)和一次签名方案. 假设$\text{ABOLF} = \{\mathsf{Gen}, \mathsf{Eval}\}$是一个$(X, Z, \tau)$-全除一有损函数, 分支空间为$B = \{0,1\}^d$. $\text{OTS} = (\mathsf{Setup}, \mathsf{Gen}, \mathsf{Sign}, \mathsf{Verify})$是一个强不可伪造一次签名方案, 其中验证密钥空间满足$VK \subseteq B$, 签名空间为$\Sigma$, $Y = B \times Z \times \Sigma$. 令$n = \log |X|$, $\tau = \log |Z|$. 下面构造一个从$X$到$Y$的不可延展函数.    
\begin{construction}[随机化的不可延展函数]~\label{con:R-NMF}
\begin{itemize} 
\item $\mathsf{Gen}(1^\kappa)$: 输入安全参数$\kappa$, 输出$s \leftarrow \text{ABOLF}.\mathsf{Gen}(1^\kappa, 0^d)$. 

\item $\mathsf{Eval}(s, x)$: 输入函数索引$s$和原像$x \in X$, 执行以下步骤:
\begin{enumerate}
\item 生成一对一次签名密钥$(vk, sigk) \leftarrow \text{OTS}.\mathsf{Gen}(1^\kappa)$;

\item 计算$z \leftarrow g_{s, vk}(x)$, $\sigma \leftarrow \text{OTS}.\mathsf{Sign}(sigk, z)$;

\item 输出$y = (vk, z, \sigma)$. 
\end{enumerate} 

\item $\mathsf{Vefy}(s, x, y)$: 输入$s$, $x$和$y = (vk, z, \sigma)$, 如果$z = g_{s, vk}(x) \wedge \text{OTS}.\mathsf{Verify}(vk, z, \sigma) = 1$, 输出``1'', 否则输出``0''.     
\end{itemize}
\end{construction}


\begin{theorem}
令$\mathcal{H}: X \rightarrow K=\{0, 1\}^\ell$是$\mathcal{F}$的一个hardcore函数族. 则$\mathcal{F}$是一个$\mathcal{H}$-hinting $\Phi$-不可延展函数族, 其中$\Phi = \Phi^{\mathsf{poly}(d)} \backslash \mathsf{cf}$, $\log d \leq n-\tau-\ell-\omega(\log \lambda)$.  
\end{theorem}

\begin{proof}
令$S_i$表示敌手在$\mathcal{A}$在$\text{Game}_i$中成功的事件.以游戏序列的方式组织证明如下:
\begin{trivlist}
\item $\text{Game}_0$: 该游戏是标准的hinting不可延展性游戏. 挑战者$\mathcal{CH}$与敌手$\mathcal{A}$按如下方式执行游戏. 
\begin{enumerate}
	\item 初始化: $\mathcal{CH}$通过运行$s \leftarrow \text{ABOLF}.\mathsf{Gen}(1^\kappa, 0^d)$生成$\mathcal{F}$的一个随机索引$s$, 并选择$h \sample \mathcal{H}$, 将$(s, h)$发送给$\mathcal{A}$. 

    \item 挑战: $\mathcal{CH}$选择$x^* \sample X$, 生成$(vk^*, sigk^*) \leftarrow \text{OTS}.\mathsf{Gen}(1^\kappa)$, 计算$z^* \leftarrow g_{s, vk^*}(x^*)$, $\sigma^* \leftarrow \text{OTS}.\mathsf{Sign}(sk^*, z^*)$, 将$(y^* = (vk^*, z^*, \sigma^*), h(x^*))$发送给$\mathcal{A}$, 其中$y^*$是函数的像值, $h(x^*)$hint函数值. 

	\item 攻击: $\mathcal{A}$输出一对元素$(\phi, y = (vk, z, \sigma))$. 如果$\mathcal{F}.\mathsf{Verify}(s, \phi(x^*), y) = 1$, 即$z = g_{s, vk}(\phi(x^*))$, $\text{OTS}.\mathsf{Verify}(vk, z, \sigma) = 1$, 则$\mathcal{A}$成功.
\end{enumerate}
根据定义, 我们有: 
\begin{equation*}
	\mathsf{Adv}_\mathcal{A} = \Pr[S_0]
\end{equation*}

\item $\text{Game}_1$: 该游戏与$\text{Game}_0$的唯一区别是$\mathcal{A}$的攻击结果$(\phi, y^*)$成功的条件定义为$\phi(x^*) = x^* \wedge \phi \in \Phi \backslash \{\mathsf{id}\}$. 由于$g_{s, vk^*}(\cdot)$是一个单射函数, $z^*$的值完全决定了它的原像, 所以敌手成功的定义仅是一种概念上的变化. 因此, 我们有
\begin{equation*}
	\Pr[S_0] = \Pr[S_1].
\end{equation*}   

\item $\text{Game}_2$: 该游戏与$\text{Game}_1$的唯一区别在于挑战者在初始化阶段就生成$(vk^*, sigk^*)$并且$s$的生成方式由$\text{ABOLF}.\mathsf{Gen}(1^\lambda, vk^*)$代替$\text{ABOLF}.\mathsf{Gen}(1^\lambda, 0^d)$. 如果存在一个敌手在两个游戏中的视图是可以区分的, 那么我们可以构造一个归约算法来攻破ABOLF的隐藏分支性质. 因此, 我们有: 
\begin{equation*}
	|\Pr[S_1] - \Pr[S_2]| \leq \mathsf{negl}(\kappa)
\end{equation*}
下面重点分析概率$\Pr[S_2]$. 在$\text{Game}_2$中, 当下列条件之一成立, 则$\mathcal{A}$的攻击结果$(\phi, x)$成功
\begin{itemize} \itemsep 1pt \parskip 0pt \parsep 0pt
	\item $E_1$: $y = y^*$且$\phi(x^*) = x^* \wedge \phi \in \Phi \backslash \mathsf{id}$. 

	\item $E_2$: $vk \neq vk^*$且$\text{OTS}.\mathsf{Vefy}(vk, z, \sigma) = 1 \wedge 
		g_{s, vk}(\phi(x^*)) = z \wedge \phi \in \Phi$. 

	\item $E_3$: $vk = vk^* \wedge (z, \sigma) \neq (z^*, \sigma^*)$且$\text{OTS}.\mathsf{Vefy}(vk^*, z, \sigma) = 1 \wedge g_{s, vk^*}(\phi(x^*)) = z \wedge \phi \in \Phi$. 
\end{itemize}
显然, $S_2 = E_1 \vee E_2 \vee E_3$. 下面分析概率$\Pr[E_i]$的上界, 其中$1 \leq i \leq 3$. 

值得注意的是$\mathcal{A}$在输出$(\phi, y)$前的视图信息是$(s, h, y^* = (vk^*, z^*, \sigma^*), h(x^*))$. 我们有: 
\begin{eqnarray}
	\avminentropy(x^*|view) & = & \avminentropy(x^*|(z^*, \sigma^*, h(x^*))) \label{eqn:ABOLF-NMF-1}\\  
							& = & \avminentropy(x^*|(z^*, h(x^*))) \label{eqn:ABOLF-NMF-2}\\	 
						   & \geq & \minentropy(x^*) - \tau - \ell  = n - \tau - \ell \label{eqn:ABOLF-NMF-3}  
\end{eqnarray}

在上述推导过程中, 公式~\eqref{eqn:ABOLF-NMF-1}源于$s$, $h$和$vk^*$与$x^*$独立这一事实. 公式~\eqref{eqn:ABOLF-NMF-2}源于$\sigma^*$是从$sk^*$和$z^*$导出且$sk^*$与$x^*$独立这一事实. 
在$\text{Game}_2$中, $g_{s, vk^*}(\cdot)$是一个有损函数, 其像空间尺寸最大为$2^\tau$. 公式~\eqref{eqn:ABOLF-NMF-3}源于引理~\ref{lemma:chain-rule}和$(z^*, h(x^*))$最多有$2^{\tau+\ell}$可能取值这一事实. 

由于$\avminentropy(x^*|view) \geq n - \tau - \ell$以及变换函数$\phi \in \Phi \backslash \mathsf{id}$的IOCR性质(参见文献~\cite{ChenQZDC22}的引理4.2), 我们有$\Pr[E_1] \leq 1/2^{n- \tau - \ell - \log d}$.   

根据$\phi \in \Phi$的HOE性质(参见文献~\cite{ChenQZDC22}的引理4.2), 我们有$\avminentropy(\phi(x^*)|view) \geq n- \tau - \ell -\log d$. 对于所有$vk \neq vk^*$, $g_{s, vk}(\cdot)$是一个单射函数, 故$z = g_{s,vk}(\phi(x^*))$的平均极小熵与$\phi(x^*)$一样. 因此, 我们有$\Pr[E_2] \leq 1/2^{n- \tau - \ell - \log d}$.

由于选择的参数$d$满足$\omega(\lambda) \leq n - \tau - \ell - \log d$, 所以我们有$\Pr[E_1]$和$\Pr[E_2]$关于安全参数$\kappa$都是可忽略的. 根据一次签名的强不可伪造性, 我们有$\Pr[E_3] \leq \mathsf{Adv}_\mathcal{A}^\text{OTS} \leq \mathsf{negl}(\lambda)$. 

通过上述分析, 可得$\Pr[S_2]$关于安全参数$\kappa$是可忽略的.      	  
\end{trivlist}

综上, 定理得证! \qed.   
\end{proof}

\begin{note}
$\mathcal{F}$的hardcore函数族可以是一个定义在$X$到$\{0, 1\}^\ell$的通用哈希函数族.
\end{note}

\subsubsection{不可延展函数的应用}
2015年, Qin等人~\cite{Qin-PKC-2015}将不可延展密钥密钥派生函数(Non-Malleable Key Derivation Functions, NM-KDFs)~\cite{FaustMVW14}推广为连续不可延展密钥派生函数(Continuous Non-Malleable KDFs, CNM-KDFs), 并提出一种利用连续不可延展密钥派生函数设计抗相关密钥攻击的密码原语(包括公钥加密, 数字签名, 基于身份加密等)的通用模式. 2022年, Chen等人~\cite{CQZDC-JOC-2022}进一步简化了此概念的名称, 称之为抗相关密钥攻击的可认证密钥派生函数(Authenticated KDFs, 简称AKDFs), 使其名称更能展现出它的性质, 并且增强了CNM-KDFs的安全性. 因此, 以AKDFs为跳板, 可以得出设计出性能良好、变换函数集范围广的多种抗相关密钥攻击的密码原语. 图~\ref{fig:constructions-applications}展示了不可延展函数的构造及其在RKA安全密码原语方面的应用.
\begin{figure}
\begin{center}
\begin{tikzpicture}
\node [rectanglenode, minimum height=2.5em, minimum width=5em, name = hintNMF] {hinting \\$\Phi$-NMF}; 
\node [textnode, name = ATDF, above of = hintNMF, xshift=-5em, yshift=5em] {ATDF};
\node [textnode, name = ABOLF, above of = hintNMF, xshift=5em, yshift=5em] {ABOLF+OTS};
\draw (ATDF) edge[->] node[left] {$\Phi_\text{brs}^\text{srs} \cup \mathsf{id}$} (hintNMF); 
\draw (ABOLF) edge[->] node[right] {$\Phi^{\mathsf{poly}(d)} \backslash \mathsf{cf}$} (hintNMF); 

\node [textnode, minimum height=1.5em, minimum width=6em, name = AKDF, right of = hintNMF, xshift=10em] {$\Phi \cup \mathsf{cf}$-RKA-secure\\AKDF};
\draw (hintNMF) edge[->] node[above] {twist} (AKDF); 

\node [rectanglenode, rounded corners=0.5em, minimum height=2.5em, minimum width=7em, name = primitives, 
	   right of = hintNMF, xshift=20em, yshift=0em] {RKA-secure \\primitives};  

\draw (primitives.west) edge[-] ($(primitives.west)+(-1.5em, 0em)$); 
\draw ($(primitives.west)+(-1.5em, 0em)$) edge[-] ($(primitives.west)+(-1.5em, 3em)$);

\draw ($(primitives.west)+(-1.5em, 3em)$) edge[-] 
	node[above] {built-in resilience \\against copy attacks} node[below] {$\Phi \subseteq \Phi_\text{brs}^\text{srs} \cup \mathsf{id} \cup \mathsf{cf}$} 
	($(primitives.east)+(1.5em, 3em)$);

\draw ($(primitives.east)+(1.5em, 3em)$) edge[-] ($(primitives.east)+(1.5em, 0em)$); 
\draw ($(primitives.east)+(1.5em, 0em)$) edge[->] (primitives.east); 
\end{tikzpicture}
\end{center}
\caption{不可延展函数的构造及应用.}
\label{fig:constructions-applications}
\end{figure}

下面, 我们重点回顾AKDFs的形式化定义及基于不可延展函数构造的AKDFs构造.
\begin{definition}[可认证密钥派生函数]
一个可认证密钥派生函数$\text{AKDFs}$包含以下三个多项式时间算法: 
\begin{itemize}\itemsep 1pt \parskip 0pt \parsep 0pt
	\item $\mathsf{Setup}(1^\kappa)$: 输入安全参数$\kappa$, 输出系统参数$pp$, 并定义了原始密钥空间$X$, 认证标签空间$T$和派生密钥空间$K$. 

	\item $\mathsf{Sample}(pp)$: 输入公开参数$pp$, 选择一个原始密钥$x \sample X$, 计算它的认证标签$t \in T$, 输出$(x, t)$. 

	\item $\mathsf{Derive}(x, t)$: 输入原始密钥$x \in X$和标签$t \in T$, 输出一个派生密钥$k \in K$或者一个拒绝符号$\bot$, 表示$t$不是$x$的合法标签. 
\end{itemize}
\end{definition}

\begin{trivlist}
\item \textbf{相关密钥攻击安全性.} 令$\Phi$是一个定义在原始密钥空间$X$上的一个变换函数集. 假设$\Phi$包含单位变换$\mathsf{id}$. 定义AKDFs敌手的优势函数如下:  
\begin{displaymath}
	\mathsf{Adv}_{\mathcal{A}, \text{AKDF}}^{\text{rka}}(\kappa) =
		\Pr \left[\beta'=\beta:~
		\begin{array}{ll}
         	& pp \leftarrow \mathsf{Setup}(1^\kappa);\\
			& (x^*, t^*) \leftarrow \mathsf{Sample}(pp);\\
			& k_0^* \leftarrow \mathsf{Derive}(x^*, t^*), k_1^* \sample K; \\
			& \beta \sample \{0,1\};\\
			& \beta' \leftarrow \mathcal{A}^{\mathcal{O}_{\mathsf{derive}}^{\Phi}}(pp, t^*, k_\beta^*);
		\end{array} 
		\right] - \frac{1}{2}.
\end{displaymath}
对于任意PPT敌手$\mathcal{A}$, 如果优势函数$\mathsf{Adv}_{\mathcal{A}, \text{AKDF}}^{\text{rka}}(\kappa)$关于安全参数$\kappa$是可忽略的, 那么称AKDFs是$\Phi$-RKA-安全的. 其中$\mathcal{O}_{\mathsf{derive}}^{\Phi}$是相关密钥派生谕言机, 输入$\langle \phi, t \rangle \neq \langle \mathsf{id}, t^* \rangle$, 返回$\mathsf{Derive}(\phi(x^*), t)$. 在实验中,  $\mathcal{A}$可以自适应地询问谕言机$\mathcal{O}_{\mathsf{derive}}^{\Phi}$. 但是, 敌手不能进行形如将$\langle \mathsf{id}, t^* \rangle$的非法询问, 否则敌手必定成功了. 根据$\phi \in \mathsf{cf}$还是$\phi \in \Phi$, 合法询问$\langle \phi, t \rangle \neq \langle \mathsf{id}, t^* \rangle$可以进一步分为常量询问和非常量询问.  
\end{trivlist}

在证明一个AKDFs方案的RKA安全性时, 首要任务在不知道原始密钥$x^*$的情况下如何回答敌手的相关密钥派生询问. 一种简单粗暴的方式是与其设法回答敌手的相关密钥派生询问, 不如直接拒绝回答敌手的所有RKA询问. 我们希望即时敌手在看到挑战信息$(x^*, t^*)$的情况下, 也无法生成一个合法的询问$\langle \phi, t \rangle$, 使得$t$是$\phi(x^*)$的合法认证标签. 因此, 在回答敌手的相关密钥派生询问时, 挑战者(模拟者)直接返回$\bot$即可. 下面介绍如何利用不可延展函数巧妙地构造RKA安全的可认证密钥派生函数.
\begin{construction}[基于不可延展函数的密钥派生函数]~\label{con:AKDFs}
令$\mathcal{F} = (\mathsf{Gen}, \mathsf{Eval}, \mathsf{Verify})$是一个$\mathcal{H}$-hinting $\Phi$-不可延展函数族, 其中$\mathcal{H}: X \rightarrow K$是一个hardcore函数, $\Phi$是一个包含单位变换$\mathsf{id}$的变换集.   
\begin{itemize}\itemsep 1pt \parskip 0pt \parsep 0pt
	\item $\mathsf{Setup}(1^\kappa)$: 输入安全参数$\kappa$, 运行$f \leftarrow \mathcal{F}.\mathsf{Gen}(1^\kappa)$, 选择一个hardcore函数$h \leftarrow \mathcal{H}$, 输出系统参数$pp = (f, h)$. 

	\item $\mathsf{Sample}(pp)$: 输入公开参数$pp = (f, h)$, 随机选择一个原始密钥$x \sample X$, 计算$t \leftarrow f(x)$, 输出$(x, t)$. 

	\item $\mathsf{Derive}(x, t)$: 输入原始密钥$x$和标签$t$, 如果$\mathcal{F}.\mathsf{Vefy}(f, x, t) = 1$成立, 则输出$k \leftarrow h(x)$, 否则输出$\bot$.   
\end{itemize}
\end{construction}

\begin{theorem}
如果$\mathcal{F}$是$\mathcal{H}$-hinting $\Phi$-不可延展的, 那么上面构造的AKDFs是$\Phi'$-RKA安全的, 其中$\Phi' = \Phi \cup \mathsf{cf}$. 
\end{theorem}

\begin{proof}
令$S_i$表示事件``敌手在$\text{Game}_i$中成功''. 以游戏序列的方式组织证明如下:
\begin{trivlist}
\item $\text{Game}_0$: 该游戏是AKDFs的标准RKA安全性游戏. 挑战者$\mathcal{CH}$与敌手$\mathcal{A}$按如下方式执行游戏:
\begin{enumerate}\itemsep 1pt \parskip 0pt \parsep 0pt
\item 初始化: $\mathcal{CH}$生成函数索引$f \leftarrow \mathcal{F}.\mathsf{Gen}(1^\kappa)$并选择一个相应的hardcore函数$h \leftarrow \mathcal{H}$, 然后将$pp = (f, h)$发送给$\mathcal{A}$. 

\item 挑战: $\mathcal{CH}$随机采样一个原始密钥$x^* \sample X$, 计算$t^* \leftarrow f(s^*)$, $k_0^* \leftarrow h(x^*)$, 选择$k_1^* \sample K$, $\beta \sample \{0,1\}$, 
	然后将$(pp, t^*, k_\beta^*)$作为挑战信息发送给$\mathcal{A}$. 

\item 相关密钥派生询问: 当收到合法询问$\langle \phi, t \rangle \neq \langle \mathsf{id}, t^* \rangle$时, 如果$\mathcal{F}.\mathsf{Vefy}(f, \phi(x^*), t) = 1$成立, $\mathcal{CH}$返回$h(\phi(x^*))$, 否则返回$\bot$. (注: 对于常量查询, $\mathcal{CH}$不需要利用$x^*$就可以进行回答.)

\item 猜测: $\mathcal{A}$输出一个猜测比特$\beta'$, 如果$\beta' = \beta$成立, 则$\mathcal{A}$成功.     
\end{enumerate}

\item 根据RKA安全性的定义, 我们有: 
\begin{equation*}
\mathsf{Adv}_{\mathcal{A}, \text{AKDF}}^{\text{rka}}(\kappa) = |\Pr[S_0] - 1/2|
\end{equation*}

\item $\text{Game}_1$(拒绝所有非常量询问): 该游戏与$\text{Game}_0$的唯一不同之处在于处理非常量询问的方式.对于所有非常量询问$\langle \phi, t \rangle$, 其中$\phi \in \Phi$,  $\mathcal{CH}$直接返回$\bot$. 令$E$为事件存在$\mathcal{A}$的非常量询问$\langle \phi, t \rangle$满足条件$\mathcal{F}.\mathsf{Verify}(f, \phi(x^*), t) = 1$. 根据$\text{Game}_0$和$\text{Game}_1$的定义, 如果事件$E$发生, 在$\text{Game}_1$中, $\mathcal{CH}$直接返回$\bot$, 而在$\text{Game}_0$中, $\mathcal{CH}$返回$\phi(x^*)$. 显而易见, 对于任意PPT敌手$\mathcal{A}$, 当事件$E$未发生时, $\mathcal{A}$在$\text{Game}_0$和$\text{Game}_1$中的视图是完全一样的. 根据差分引理, 我们有: 
    \begin{equation*}
        |\Pr[S_0] - \Pr[S_1]| \leq \Pr[E]
    \end{equation*} 
	
根据下面的引理, 当$\mathcal{F}$满足$\mathcal{H}$-hinting $\Phi$-不可延展性时, 事件$E$发生的概率是可忽略的.
\begin{claim}\label{lemma:AKDF-hinting-NM}
	$\Pr[E] \leq \mathsf{poly}(\kappa)\cdot \mathsf{Adv}_{\mathcal{A}, \mathcal{F}}^{\Phi-\text{nm}}(\kappa)$. 
\end{claim} 

\begin{proof}
令$\mathcal{B}$是一个攻击$\mathcal{F}$的$\mathcal{H}$-hinting $\Phi$-不可延展性的敌手. 给定挑战信息$(f, h, y^*, h(x^*))$, 其中$f \leftarrow \mathcal{F}.\mathsf{Gen}(1^\kappa)$, $h \leftarrow \mathcal{H}$, $y^* \leftarrow f(x^*)$并且$x^* \sample X$, $\mathcal{B}$按以下方式模拟$\mathcal{A}$在$\text{Game}_1$中的挑战者: 令$pp = (f, h)$, $t^* = y^*$, $k_0^* \leftarrow h(x^*)$, 选择$k_1^* \sample K$, $\beta \sample \{0,1\}$, 然后将$(pp, t^*, k_\beta^*)$发送给$\mathcal{A}$. 这里, $\mathcal{B}$并不知道$x^*$的值. 然而, 这并不影响$\mathcal{B}$模拟回答$\mathcal{A}$的询问, 因为在$\text{Game}_1$中, 对于所有非常量查询, $\mathcal{B}$直接返回$\bot$. 由此可知, $\mathcal{B}$能完美地模拟$\mathcal{A}$在$\text{Game}_1$中的视图环境. 令$L$为$\mathcal{A}$的所有非常量询问列表. 由于$\mathcal{A}$是一个PPT敌手, 所以$|L| \leq \mathsf{poly}(\kappa)$. 最后, $\mathcal{B}$从列表$L$中随机选择一组元素$\langle \phi, t\rangle$作为攻击$\mathcal{F}$ $\mathcal{H}$-hinting $\Phi$-不可延展性的结果. 当事件$E$发生时, $\mathcal{B}$成功的概率至少是$1/\mathsf{poly}(\kappa)$. 因此, $\mathcal{B}$成功的优势至少为$\Pr[E]/\mathsf{poly}(\kappa)$. 如果$\Pr[E]$是不可忽略的, 那么$\mathcal{B}$的优势也是不可忽略的, 与$\mathcal{F}$的不可延展性矛盾. 特别地, 由于$\mathcal{B}$攻击$\mathcal{F}$不可延展性成功的概率不超过$\mathsf{Adv}_{\mathcal{A}, \mathcal{F}}^{\Phi-\text{nm}}(\kappa)$, 所以 $\Pr[E]/\mathsf{poly}(\kappa) \leq \mathsf{Adv}_{\mathcal{A}, \mathcal{F}}^{\Phi-\text{nm}}(\kappa)$. 引理得证! \qed 
\end{proof}

\item $\text{Game}_2$ ($k_0^* \sample K_1$): 该游戏与$\text{Game}_1$的唯一不同之处在于$\mathcal{CH}$随机选择$k_0^* \sample K$, 而不是通过$k_0^* \leftarrow h(x^*)$计算而来. 显然, 如果存在一个敌手$\mathcal{A}$在$\text{Game}_1$和$\text{Game}_2$中的视图存在差异$\epsilon(\kappa)$, 我们可以构造一个归约算法以至少$\epsilon(\kappa)/2$的优势攻破hardcore函数的伪随机性. 因此, 我们有: 
    \begin{equation*}
        |\Pr[S_1] - \Pr[S_2]| \leq \mathsf{negl}(\kappa)
    \end{equation*} 

在$\text{Game}_2$中, $k_\beta$的值与$\beta$完全独立. 故, $\Pr[S_2] = 1/2$.  
\end{trivlist}

综上, 定理得证! \qed 
\end{proof} 


