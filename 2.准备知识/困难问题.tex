\section{困难问题}
密码学中常见的困难问题可大致分为数论类和格类, 其中前者是代数问题, 后者是数的几何问题, 
这也正是格类困难问题具备抗量子攻击的原因之一. 

数论类的假设又可进一步分为整数分解类和离散对数类两个分支. 本章首先介绍常见的数论类假设, 再介绍常见的格类困难问题. 

\subsection{整数分解类假设}
整数分解类假设定义在群$\mathbb{Z}_N^*$上. 其中:
\begin{equation*}
\mathbb{Z}_N^* \define \{b \in \{1, \dots, N_1\} \mid \text{gcd}(b, N)=1\}
\end{equation*}
也即$\mathbb{Z}_N^*$是整数集合$\{1, \dots, N-1\}$中所有与$N$互质元素构成的子集, 
在模乘运算$ab \define [ab \bmod N]$下构成交换群.  

以下令$\mathsf{GenModulus}$是多项式时间算法, 其以安全参数$1^\kappa$为输入, 输出$(N=pq, p, q)$, 
其中$p$和$q$(以压倒性概率)是两个$\kappa$-bit的素数. 

\begin{definition}[整数分解假设]
整数分解问题指分解大整数在平均意义下是困难的. 我们称\emph{整数分解假设}相对于$\mathsf{GenModulus}$成立当且仅当对于任意PPT敌手: 
\begin{equation*}
\Pr[\mathcal{A}(N) = (p', q') \text{~s.t.~} p'q' = N] \leq \mathsf{negl}(\kappa)
\end{equation*} 	
上述概率建立在敌手$\mathcal{A}$和$\mathsf{GenModulus}(1^\kappa) \rightarrow (N, p, q)$的随机带上. 
\end{definition}

尽管整数分解假设历经百年的分析攻击仍然健壮, 但是它并不直接蕴含高效的密码系统. 
这就引发密码科技工作者研究与整数分解问题困难性相关问题的研究. 1978年, Rivest, Shamir和Adleman提出了RSA问题. 

令$\mathsf{GenRSA}$是多项式时间算法, 其以安全参数$1^\kappa$为输入, 
输出两个$\kappa$-bit素数的乘积$N$作为模数, 同时输出正整数$(e, d)$满足$ed = 1 \bmod \phi(N)$. 

\begin{definition}[RSA假设]
RSA问题指在平均意义下求解$\mathbb{Z}_N^*$的$e$次方根是困难的. 
我们称\emph{RSA假设}相对于$\mathsf{GenModulus}$成立当且仅当对于任意PPT敌手:
\begin{equation*}
\Pr[\mathcal{A}(N, e, y) = x \text{~s.t.~} x^e = y \bmod N] \leq \mathsf{negl}(\kappa)
\end{equation*} 	
上述概率建立在敌手$\mathcal{A}$、$\mathsf{GenRSA}(1^\kappa) \rightarrow (N, e, d)$和随机选取$x \in \mathbb{Z}_N^*$随机带上. 
\end{definition}

\begin{remark}
RSA问题刻画的是在不知晓$\phi(N)$的情况下计算$\mathbb{Z}_N^*$中随机元素的$e$次方根是困难的. 
容易看出, 如果敌手能够打破整数分解问题, 则可以通过分解$N$求出$\phi(N)$, 进而通过Fermat小定理计算$e$次方根. 
因此, 整数分解问题难于RSA问题, 整数分解假设弱于RSA假设. 
两个假设是否等价仍然未知. 
\end{remark}

给定群$\mathbb{G}$, 称$y \in \mathbb{G}$是一个二次剩余当且仅当$\exists x \in \mathbb{G}$ s.t. $x^2 = y$. 
$x$成为$y$的平方根. 如果一个元素不是二次剩余则称其为二次非剩余. 
在abelian群中, 二次剩余构成子群.

首先考察群$\mathbb{Z}_p^*$中的二次剩余, 其中$p$是素数. 
定义函数$\mathsf{sq}_p: \mathbb{Z}_p^* \rightarrow \mathbb{Z}_p^*$为$\mathsf{sq}_p(x) \define [x^2 \bmod p]$. 
当$p$是大于2的素数时, $\mathsf{sq}_p$是2-对-1函数, 因此立刻可知$\mathbb{Z}_p^*$中恰好一半元素是二次剩余. 
记模$p$的二次剩余集合为$\mathcal{QR}_p$, 模$p$的二次非剩余集合为$\mathcal{QNR}_p$, 我们有: 
\begin{equation*}
|\mathcal{QR}_p| = |\mathcal{QNR}_p| = \frac{|\mathbb{Z}_p^*|}{2} = \frac{p-1}{2}
\end{equation*}

定义元素$x \in \mathbb{Z}_p^*$模$p$的Jacobi符号如下: 
\begin{displaymath}
\mathcal{J}_p(x) \define
\left\{
\begin{array}{l}
+1  \text{~if~} x \in \mathcal{QR}_p\\
+1 \text{~if~} x \in \mathcal{QNR}_p\\
\end{array}
\right.
\end{displaymath}

再考察群$\mathbb{Z}_N^*$中的二次剩余, 其中$N$是两个互异素数$p$和$q$的乘积. 
由中国剩余定理可知: $\mathbb{Z}_N^* \simeq \mathbb{Z}_p^* \times \mathbb{Z}_q^*$, 
令$y \leftrightarrow (y_p, y_q)$表示上述同构映射给出的分解, 易知$y$是模$N$的二次剩余当且仅当$y_p$和$y_q$分别是模$p$和模$q$的二次剩余. 
定义函数$\mathsf{sq}_N: \mathbb{Z}_N^* \rightarrow \mathbb{Z}_N^*$为$\mathsf{sq}_N(x) \define [x^2 \bmod N]$, 
当$N$为互异素数乘积时, $\mathsf{sq}_N$是4--对--1函数. 
记模$N$的二次剩余集合为$\mathcal{QR}_N$, 由$\mathcal{QR}_N$与$\mathcal{QR}_p \times \mathcal{QR}_q$之间的一一对应关系可知: 

\begin{equation*}
\frac{|\mathcal{QR}_N|}{|\mathbb{Z}_N^*|} = \frac{|\mathcal{QR}_p| \cdot |\mathcal{QR}_q|}{|\mathbb{Z}_N^*|} 
= \frac{1}{4}
\end{equation*}

从二次剩余的角度可以对$\mathbb{Z}_N^*$中的元素进行如下的划分: 
(i) $\mathbb{Z}_N^*$可以划分为相同大小的$\mathcal{J}_N^{+1}$和$\mathcal{J}_N^{-1}$(Jacobi符号分别为1和-1); 
(ii) $\mathcal{J}_N^{+1}$有可以划分为$\mathcal{QR}_N$和$\mathcal{QNR}_N^{+1}$, 
其中$\mathcal{QNR}_N^{+1} \define \{x \in \mathbb{Z}_N^* \mid x \notin \mathcal{QR}_N \wedge \mathcal{J}_N(x)=+1\}$.  

\begin{definition}[模二次剩余假设]
模二次剩余(QR, quadratic residue)假设指$\mathcal{QR}_N$上的均匀分布与$\mathcal{QNR}_N^{+1}$上的均匀分布计算不可区分. 
我们称\emph{模二次剩余假设}相对于$\mathsf{GenModulus}$成立当且仅当对于任意PPT敌手:  
\begin{equation*}
|\Pr[\mathcal{A}(N, y_0) = 1] - \Pr[\mathcal{A}(N, y_1) = 1]| \leq \mathsf{negl}(\kappa)
\end{equation*}		
上述概率建立在敌手$\mathcal{A}$、$\mathsf{GenModulus}(1^\kappa) \rightarrow (N, p, q)$和
随机选取$y_0 \in \mathcal{QR}_N, y_1 \in \mathcal{QNR}_N^{+1}$的随机带上. 
\end{definition}

与QR假设应用紧密相关的技术细节是如何对$\mathcal{QR}_N$和$\mathcal{QNR}_N^{+1}$进行高效的均匀采样. 
\begin{itemize}
\item 对$\mathcal{QR}_N$进行均匀采样较为简单: 仅需随机选取$x \in \mathbb{Z}_N^*$再令$y:= x^2 \bmod N$即可. 
	注意到$x^2 \bmod N$是一个4-to-1的正则函数, 因此当$x \sample \mathbb{Z}_N^*$时, 输出$y$服从$\mathcal{QR}_N$上的均匀分布. 

\item 对$\mathcal{QNR}_N^{+1}$进行均匀采样稍显复杂, 当$N$的分解未知时如何均匀采样未知. 
	我们可以借助辅助信息$z \in \mathcal{QNR}_N^{+1}$完成采样, 即随机选取$x \in \mathbb{Z}_N^*$, 
	输出$y:= z \cdot x^2 \bmod N$. 可以验证, 当$x \sample \mathbb{Z}_N^*$时, 输出$y$服从$\mathcal{QNR}_N^{+1}$上的均匀分布. 
\end{itemize}


\begin{remark}
显然, 整数分解问题难于二次剩余判定问题, 因此整数分解假设弱于模二次剩余判定假设. 
两个假设是否等价仍然未知. 
\end{remark}

\begin{definition}[模平方根假设]
模平方根(SQR, square root)假设指对$\mathcal{QN}_N$中的随机元素求平方根是困难的. 
我们称\emph{模平方根假设}相对于$\mathsf{GenModulus}$成立当且仅当对于任意PPT敌手:  
\begin{equation*}
\Pr[\mathcal{A}(N, y) = x \text{~s.t.~} x^2 = y \bmod N] \leq \mathsf{negl}(\kappa)
\end{equation*}	
上述概率建立在敌手$\mathcal{A}$、$\mathsf{GenModulus}(1^\kappa) \rightarrow (N, p, q)$和
随机选取$y \in \mathcal{QR}_N$的随机带上. 
\end{definition}

令$p$和$q$是两个互异的模4余3的素数, 则称$N = pq$是Blum整数. 我们有以下推论: 
\begin{proposition}
当$N$时Blum整数时, 则每个模$N$的二次剩余有且仅有一个平方根是二次剩余.
\end{proposition}
上述推论保证了当$N$是Blum整数时, 函数$f_N \define [x^2 \bmod N]$构成$\mathcal{QR}_N$上的置换. 这一性质在构造加密方案时至关重要. 

\begin{remark}
模平方根假设等价于整数分解假设, 即在未知$N$分解的情况下求模平方根与分解$N$一样困难.  
\end{remark}

综上, 整数分解类问题的困难性关系如图~\ref{figure:factoring-type-problems-relation}所示: 
\begin{figure}[!hbtp]
\begin{center}
\begin{tikzpicture}
\node[textnode, name=factoring] {整数分解问题}; 
\node[textnode, right of = factoring, xshift=5em] {$\equiv$}; 
\node[textnode, name=SQR, right of = factoring, xshift=10em] {求模平方根问题}; 
\node[textnode, name=RSA, right of = SQR, anchor=west, xshift=8em, yshift=4em] {$\succeq$RSA问题};
\node[textnode, name=QR, right of = SQR, anchor=west, xshift=8em, yshift=-4em] {$\succeq$模二次剩余问题}; 
\draw (SQR.east) edge[-] (RSA.west); 
\draw (SQR.east) edge[-] (QR.west); 
\end{tikzpicture}
\end{center}
\caption{整数分解类问题的困难性关系}\label{figure:factoring-type-problems-relation}
\end{figure}

\subsection{离散对数类假设}
离散对数类假设定义在循环群$\mathbb{G}$中. 
令$\mathsf{GenGroup}$是多项式时间算法, 其以安全参数$1^\kappa$为输入, 输出$q$阶循环群$\mathbb{G} = \langle g \rangle$的描述, 
其中, $q$是$\kappa$-bit的整数, 简记为$(\mathbb{G}, q, g) \leftarrow \mathsf{GenGroup}(1^\kappa)$. 
为了行文方便, 本书中假设$\mathbb{G}$为加法群, 用``$\cdot$''表示群运算.  
由循环群的定义可知, $\mathbb{G}$中的元素为$\{g^0, g^1, \dots, g^{q-1}\}$. 因此, 对于任意$h \in \mathbb{G}$
存在唯一的$x \in \mathbb{Z}_q$使得$g^x = h$, 我们称$x$是$h$相对于生成元$g$的离散对数并记为$x = \log_g h$, 
这里称其为离散对数强调其取值均为非负整数, 有别于标准算术对数的取值为实数.  



\begin{definition}[离散对数假设]
离散对数问题指在平均意义下求解群元素的离散对数是困难的. 
我们称\emph{离散对数(DLOG)假设}相对于$\mathsf{GenGroup}$成立当且仅当对于任意PPT敌手: 
\begin{equation*}
\Pr[\mathcal{A}(\mathbb{G}, q, g, h) = \log_g h] \leq \mathsf{negl}(\kappa)
\end{equation*} 
上述概率建立在敌手$\mathcal{A}$、$\mathsf{GenGroup}(1^\kappa) \rightarrow (\mathbb{G}, q, g)$和随机采样$h \in \mathbb{G}$的随机带上. 
\end{definition}

显然, 离散对数假设说明了$x \mapsto g^x$是从$\mathbb{Z}_q$到$\mathbb{G}$的单向函数. 
单向函数能够蕴含的密码方案有限, 下面介绍与离散对数假设相关的其它假设, 它们能够作为更多密码方案的安全基础. 
这类困难假设的起源于Diffie和Hellman~\cite{DH-IEEE-IT-1976}在1976年的划时代论文, 
后来被称为Diffie-Hellman假设. 
为了叙述方便, 我们首先定义DH函数$\mathsf{DH}_g: \mathbb{G}^2 \rightarrow \mathbb{G}$, 
\begin{equation*}
\mathsf{DH}_g(h_1, h_2) \define g^{\log_g h_1 \cdot \log_g h_2}
\end{equation*} 

% \begin{equation*}
% (g^a)^b = g^{ab} = (g^b)^a
% \end{equation*}

% \begin{equation*}
% g^{\red{x}} = y
% \end{equation*}

% \begin{equation*}
% \red{m}^e = c
% \end{equation*}

Diffie-Hellman类假设可细分为两类, 一类是计算性Diffie-Hellman (CDH)问题, 一类是判定性Diffie-Hellman (DDH)问题. 
下面依次介绍.

\begin{definition}[CDH假设]
CDH问题指在平均意义下计算$\mathsf{DH}_g$函数是困难的. 
我们称\emph{CDH假设}相对于$\mathsf{GenGroup}$成立当且仅当对于任意PPT敌手: 
\begin{equation*}
\Pr[\mathcal{A}(\mathbb{G}, q, g, g^a, g^b) = g^c] \leq \mathsf{negl}(\kappa)
\end{equation*} 	
上述概率建立在敌手$\mathcal{A}$、$\mathsf{GenGroup}(1^\kappa) \rightarrow (\mathbb{G}, q, g)$和随机采样$a, b \in \mathbb{Z}_q$的随机带上. 
\end{definition}  

\begin{definition}[DDH假设]
对于四元组$(g, g^a, g^b, g^c)$, 如果$g^c = \mathsf{DH}_g(g^a, g^b)$也即$ab = c \bmod q$, 则称其为DH元组. 
DDH假设刻画的是随机DH元组和随机四元组是计算不可区分的. 我们称\emph{DDH假设}相对于$\mathsf{GenGroup}$成立当且仅当对于任意PPT敌手: 
\begin{equation*}
|\Pr[\mathcal{A}(\mathbb{G}, q, g, g^a, g^b, g^{ab})=1] - \Pr[\mathcal{A}(\mathbb{G}, q, g, g^a, g^b, g^c)=1]| 
\leq \mathsf{negl}(\kappa)
\end{equation*}  
上述概率建立在敌手$\mathcal{A}$、$\mathsf{GenGroup}(1^\kappa) \rightarrow (\mathbb{G}, q, g)$和随机采样$a, b, c \in \mathbb{Z}_q$的随机带上. 
\end{definition} 

离散对数类问题的困难性关系如图~\ref{figure:dlog-type-problems-relation}所示: 
\begin{figure}[!hbtp]
\begin{center}
\begin{tikzpicture}
\node[textnode, name=dlog] {离散对数问题}; 
\node[textnode, right of = dlog, xshift=5em] {$\succeq$}; 
\node[textnode, name=CDH, right of = dlog, xshift=10em] {CDH问题}; 
\node[textnode, right of = CDH, xshift=5em] {$\succeq$}; 
\node[textnode, name=DDH, right of = CDH, xshift=10em] {DDH问题}; 
\end{tikzpicture}
\end{center}
\caption{离散对数类问题的困难性关系}\label{figure:dlog-type-problems-relation}
\end{figure}

\begin{remark}
注意到任何$q$阶循环群均与$\mathbb{Z}_q$是同构的, 而$\mathbb{Z}_q$上的离散对数问题时容易的. 
因此在实例化循环群$\mathbb{G}$必须谨慎审慎, 这也从一个方面说明离散对数类问题的困难性与底层代数结构的具体特性(如群的表示)紧密相关. 
对于$\mathbb{G}$的实例化, 通常既可以选择$\mathbb{F}_{p^k}^*$的素数阶乘法子群, 
也可以选择椭圆曲线上的素数阶乘法群. 
另外强调一点, 存在这样的循环群$\mathbb{G}$(如双线性映射群)使得离散对数、CDH假设成立, 而DDH假设不成立. 
\end{remark}


\subsection{格类假设} 
1997年, Shor~\cite{Shor-SIAM-1997}给出了数论类问题(包括整数分解类和离散对数类)的有效量子算法. 
在未来, 如果大规模量子计算机研制成功, 则数论类假设将不再成立. 迄今为止, 尚未有针对格基困难问题的有效量子算法, 
通用的量子算法仅相对非量子算法有些许优势. 目前普遍的共识是格基困难问题具备抗量子安全能力, 这正是该类问题倍受关注的主要原因. 

本小节中将介绍两个主要的平均意义下的格基困难问题, 短整数解问题和带误差学习问题. 
需要提前说明的是, 格基困难问题的困难性与参数的选取密切相关, 因此格基问题的描述相比数论类问题要复杂得多. 

Ajtai~\cite{Ajtai-STOC-1996}在1996年的开创性论文中正式提出了短整数解(short integer solution, SIS)问题. 
SIS问题不仅可以作为所有Minicrypt世界中密码组件的安全基础, 包括单向函数、身份鉴别协议、数字签名, 
还可以用来构造抗碰撞哈希函数. 非正式的, SIS问题指在给定许多较大的有限加法群中随机选取的元素, 找到足够``短''的整系数组合使得其和是0是困难的. 
SIS问题由以下参数刻画:    
\begin{itemize}
	\item 正整数$n$和$q$, 用于刻画加法群$\mathbb{Z}_q^n$; 
	\item 正实数$\beta$, 用于刻画解向量的长度; 
	\item 正整数$m$, 用于表征群元素的个数. 
\end{itemize}
其中$n$是主要的参数(如: $n \geq 100$), $q > \beta$通常设定为关于$n$的小多项式. 

\begin{definition}[短整数解假设($\text{SIS}_{n, q, \beta, m}$)]
我们称\emph{SIS假设}成立当且仅当对于任意PPT敌手:
\begin{equation*}
\Pr[\mathcal{A}(\mathbf{a}_1, \dots, \mathbf{a}_m) = \mathbf{z} \neq \mathbf{0} \in \mathbb{Z}^m \text{~s.t.~} 
\sum_{i}^m \mathbf{a}_i z_i = \mathbf{0} \in \mathbb{Z}_q^n \wedge \lVert z \rVert \leq \beta] \leq \mathsf{negl}(\kappa) 
\end{equation*} 
上述概率建立在敌手$\mathcal{A}$和随机选取$\mathbf{a}_i \in \mathbb{Z}^m$的随机带上. 
\end{definition}
以上定义中$m$个$\mathbb{Z}_q^n$上的随机向量可以按列向量的方式组成矩阵$\mathbf{A} \in \mathbb{Z}_q^{n \times m}$. 
因此, SIS假设实质上在要求找到函数$f_\mathbf{A}(\mathbf{z}) := \mathbf{A} \mathbf{z}$的短整数非零向量原像是困难的. 

下面简单讨论参数选取与问题困难性之间的关联: 
\begin{itemize}
\item 如果不对$\lVert z \rVert$进行限制, 那么可以轻易利用Gaussian消元法找到一个整数解. 
	同时, 我们必须要求$\beta < q$, 否则$\mathbf{z} = (q, 0, \dots, 0) \in \mathbb{Z}^m$即构成一个合法的非平凡解. 

\item 注意到任何关于矩阵$\mathbf{A}$的短整数解可通过补$0$平凡地延展为关于矩阵$[\mathbf{A} \mid \mathbf{A}']$的解. 
	换言之, SIS问题的困难性随着$m$的增大变得容易. 对应的, SIS问题的困难性随着$n$增加变得困难. 

\item 向量范数界$\beta$和向量$\mathbf{a}_i$的个数$m$必须足够大以保证解的存在性. 
	令$\bar{m}$是大于$n \log q$的最小正整数, 则我们必须有$\beta > \sqrt{\bar{m}}$和$m \geq \bar{m}$. 
	不失一般性, 不妨假设$m = \bar{m}$, 则存在超过$q^n$个向量$\mathbf{x} \in \{0,1\}^m$, 
	根据鸽巢原理, 则必有$\mathbf{x} \neq \mathbf{x}'$使得$\mathbf{A}\mathbf{x} = \mathbf{A}\mathbf{x}' \in \mathbb{Z}_q^n$, 
	从而它们的差值$\mathbf{z} = \mathbf{x} - \mathbf{x}' \in \{0, \pm 1\}^m$是范数小于$\beta$的短整数解. 

\item 上述的鸽巢原理论证事实上蕴含更多深意: 函数族$\{f_\mathbf{A}: \{0,1\}^m \rightarrow \mathbb{Z}_q^n\}$基于SIS假设是抗碰撞的. 
	若不然, 给定关于$f_\mathbf{A}$的一对碰撞$\mathbf{x}, \mathbf{x}' \in \{0,1\}^m$, 则立刻诱导出关于$\mathbf{A}$的一个短整数解. 
\end{itemize}

SIS问题可以被理解为在以下特定$q$元$m$维整数格中的平均意义短向量问题(short-vector problem, SVP), 该整数格的定义为: 
\begin{equation*}
\mathcal{L}^{\bot}(\mathbf{A}) \define \{\mathbf{z} \in \mathbb{Z}^m: \mathbf{A}\mathbf{z} = 0 \in \mathbb{Z}_q^n\} 
\supseteq q\mathbb{Z}^m
\end{equation*}
从编码的角度理解, $\mathbb{A}$扮演着格/码字$\mathcal{L}^{\bot}(\mathbf{A})$校验矩阵的角色. 
SIS问题的困难性指对于随机选取的$\mathbf{A}$, 找到一个短的码字是困难的. 


Regev~\cite{Regev-STOC-2005}在2005年的开创性论文中提出了另一个平均意义下的重要格基困难问题--带误差学习问题(learning with errors, LWE). 
LWE问题与SIS问题互相对偶, 能够蕴含Minicrypt之外的密码体制. 

在正式定义LWE问题之前, 首先引入LWE分布的概念. 称向量$\mathbf{s} \in \mathbb{Z}_q^n$为秘密, 
LWE分布$A_{\mathbf{s}, \chi}$定义在$\mathbb{Z}_q^n \times \mathbb{Z}_q$上, 
采样算法为随机选取$\mathbf{a} \in \mathbb{Z}_q^n$, 选取$e \leftarrow \chi$, 
输出$(\mathbf{a}, b = \langle \mathbf{s}, \mathbf{a}\rangle+e \bmod q)$. 

LWE问题有两个版本, 其中搜索版本要求给定LWE采样求解秘密, 判定版本要求区分LWE采样和随机采样. 
LWE问题由以下参数刻画: 
\begin{itemize}
\item 正整数$n$和$q$, 和SIS问题一样, 用于刻画加法群$\mathbb{Z}_q^n$;
\item 正整数$m$表征采样的个数, 通常选取的足够大以保证秘密的惟一性; 
\item $\mathbb{Z}$上的误差分布$\chi$, 通常的选取是宽度为$\alpha q$的离散Gaussian分布, 其中$\alpha < 1$称为相对错误率. 
\end{itemize} 

\begin{definition}[搜索LWE假设]
搜索LWE问题指给定$m$个$A_{\mathbf{s}, \chi}$的独立随机采样, 求解秘密向量$\mathbf{s}$是困难的. 
我们称\emph{搜索LWE假设}成立当且仅当对于任意PPT敌手:
\begin{equation*}
\Pr[\mathcal{A}(\{\mathbf{a}_i, b\}_{i=1}^m \leftarrow A_{\mathbf{s}, \chi}) = \mathbf{s}] \leq \mathsf{negl}(\kappa) 
\end{equation*} 
上述概率建立在敌手$\mathcal{A}$、随机选取$\mathbf{s} \in \mathbb{Z}_q^n$和采样$A_{\mathbf{s}, \chi}$的随机带上. 
\end{definition}

\begin{definition}[判定LWE假设]
判定LWE问题指区分$m$个独立采样是来自$A_{\mathbf{s}, \chi}$分布还是随机分布是困难的. 
我们称\emph{判定LWE假设}成立当且仅当对于任意PPT敌手:
\begin{equation*}
|\Pr[\mathcal{A}(\{\mathbf{a}_i, b\}_{i=1}^m \leftarrow A_{\mathbf{s}, \chi}) = 1] - 
 \Pr[\mathcal{A}(\{\mathbf{a}_i, b\}_{i=1}^m \leftarrow U_{\mathbb{Z}_q^n \times \mathbb{Z}_q}) = 1] \leq \mathsf{negl}(\kappa) 
\end{equation*} 
上述概率建立在敌手$\mathcal{A}$、随机选取$\mathbf{s} \in \mathbb{Z}_q^n$和采样$A_{\mathbf{s}, \chi}$
以及$U_{\mathbb{Z}_q^n \times \mathbb{Z}_q}$的随机带上. 
\end{definition}

\begin{remark}
LWE问题时LPN问题(learning parities with noise)的一般化. 在LPN问题中, $q=2$, $\chi$为$\{0,1\}$上的Bernoulli分布. 
\end{remark}

下面简单讨论参数选取与问题困难性之间的关联: 
\begin{itemize}
\item 如果没有误差分布$\chi$, 则LWE问题的搜索版本和判定版本均可利用Gaussian消元法快速求解. 

\item 和SIS问题类似, 可以用矩阵的语言更简洁的描述LWE问题: 
	(i) 将$m$个向量$\mathbf{a}_i \in \mathbb{Z}_q^n$汇聚为矩阵$\mathbf{A} \in \mathbb{Z}_q^{n \times m}$; 
	(ii) 将$m$个$b_i \in \mathbb{Z}_q$汇聚为向量$\mathbf{b} \in \mathbb{Z}_q^n$, 因此对于LWE采样我们有: 
	\begin{equation*}
		\mathbf{b}^t = \mathbf{s}^t \mathbf{A} + \mathbf{e}^t (\bmod q),
	\end{equation*}
	其中$\mathbf{e} \leftarrow \chi^m$. 
\end{itemize}

LWE问题可以被理解为在以下特定$q$元$m$维整数格中的平均意义有界距离解码问题(bounded-distance decoding problem, BDD), 该整数格的定义为: 
\begin{equation*}
\mathcal{L}(\mathbf{A}) \define \{\mathbf{A}^t \mathbf{s}: \mathbf{s} \in \mathbb{Z}_q^n\}+q \mathbb{Z}^m 
\end{equation*}

从编码的角度理解, $\mathbf{A}$扮演着格/码字$\mathcal{L}(\mathbf{A})$生成矩阵的角色. 
对于LWE采样, $\mathbf{b}$与格中的惟一向量/码字相近, 搜索版本要求计算秘密向量$s$, 即根据带误差的码字进行解码. 
对于随机采样, $\mathbf{b}$以大概率远离格$\mathcal{L}(\mathbf{A})$中所有向量.   
SIS问题的困难性指对于随机选取的$\mathbf{A}$, 找到一个短的码字是困难的. 