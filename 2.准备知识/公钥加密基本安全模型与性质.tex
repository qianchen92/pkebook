\section{公钥加密基本安全模型}
\subsection{公钥加密方案}
公钥加密的概念由Diffie和Hellman~\cite{DH-IEEE-IT-1976}在1976年的划时代论文中正式提出, 
其与对称加密的最大不同在于每个用户自主生成一对密钥, 公钥用于加密、私钥用于解密, 发送方仅需知晓接收方的公钥即可向接收方发送密文. 

\begin{definition}[公钥加密方案]
正式的, 公钥加密方案由以下的四个多项式时间组成:   
\begin{itemize}
\item $\mathsf{Setup}(1^\kappa)$: 系统生成算法以安全参数$1^\lambda$为输入, 输出系统公开参数$pp$, 其中$pp$包含对
	公钥空间$PK$、私钥空间$SK$、明文空间$M$和密文空间$C$的描述. 该算法由可信第三方生成并公开, 系统中的所有用户共享, 所有算法均将$pp$作为输入.  
	当上下文明确时, 常常为了行文简洁省去$pp$. 

\item $\mathsf{KeyGen}(pp)$: 密钥生成算法以系统公开参数$pp$为输入, 输出一对公/私钥对$(pk, sk)$, 其中公钥公开, 私钥秘密保存. 

\item $\mathsf{Encrypt}(pk, m; r)$: 加密算法以公钥$pk \in PK$、明文$m \in M$为输入, 输出密文$c \in C$. 

\item $\mathsf{Decrypt}(sk, c)$: 解密算法以私钥$sk \in SK$和密文$c \in C$为输入, 
	输出明文$m \in M$或者$\bot$表示密文非法. 解密算法通常为确定性算法. 
\end{itemize}
\end{definition}

\begin{trivlist}
\item \textbf{正确性.} 该性质保证公钥加密的功能性, 即使用私钥可以正确恢复出对应公钥加密的密文. 正式的, 对于任意明文$m \in M$, 有:
\begin{equation}\label{equation:pke-correctness}
	\Pr[\mathsf{Decrypt}(sk, \mathsf{Encrypt}(pk, m)) = m] = 1 - \mathsf{negl}(\kappa).
\end{equation}
公式~\eqref{equation:pke-correctness}的概率建立在$\mathsf{Setup}(1^\kappa) \rightarrow pp$、
$\mathsf{KeyGen}(pp) \rightarrow (pk, sk)$和$\mathsf{Encrypt}(pk, m) \rightarrow c$的随机带上. 
如果上述概率严格等于1, 则称公钥加密方案满足完美正确性. 
\end{trivlist}

\begin{remark}
通常基于数论假设的公钥加密方案满足完美正确性, 而格基方案由于底层困难问题的误差属性, 解密算法存在可忽略的误差. 
\end{remark}

\begin{figure}[!hbtp]
\begin{center}
\begin{tikzpicture}
	\node [name=Setup, textnode] {$\mathsf{Setup}(1^\kappa) \rightarrow pp$}; 

    \node [name=Alice, textnode, below of = Setup, xshift=-8em, yshift=-6em] 
    	{\includegraphics[width=0.5in]{插图/alice.png}}; 
    \node [textnode, above of=Alice, yshift=3em] {Alice};
    \node [textnode, name=kA, below of=Alice, yshift=-3em] {$\mathsf{KeyGen}(pp) \rightarrow (pk_a, sk_a)$};

    \node [name=Bob, textnode, below of = Setup, xshift=8em, yshift=-6em] 
    	{\includegraphics[width=0.5in]{插图/bob.png}}; 
    \node [textnode, above of=Bob, yshift=3em] {Bob};
    \node [textnode, name=kB, below of=Bob, yshift=-3em] {$\mathsf{KeyGen}(pp) \rightarrow (pk_b, sk_b)$};

    \draw (Alice) edge[->, thick] node[auto] {$\mathsf{Encrypt}(pk_b, m) \rightarrow c$} (Bob);

    \node [name=message, textnode, right of= Bob, xshift=12em] {$m$};
    \draw (Bob) edge[->, thick] node[auto] {$\mathsf{Decrypt}(sk_b, c)$} (message);  
\end{tikzpicture}
\end{center}
\caption{公钥加密方案示意图}\label{figure:PKE-scheme}
\end{figure}



\begin{trivlist}
\item \textbf{安全性.} 定义公钥加密方案敌手$\mathcal{A} = (\mathcal{A}_1, \mathcal{A}_2)$的优势函数如下: 
\begin{displaymath}
	\AdvA(\kappa) = \Pr \left[ \beta' = \beta:~
	\begin{array}{ll}
		& pp \leftarrow \mathsf{Setup}(1^\lambda); \\		
		& (pk, sk) \leftarrow \mathsf{KeyGen}(pp);\\
		& (m_0, m_1, state) \leftarrow \mathcal{A}_1^{\blue{\Odecrypt(\cdot)}}(pp, pk); \\
		& \beta \sample \{0,1\};\\ 
        & c^* \leftarrow \mathsf{Encrypt}(pk, m_\beta);\\
        & \beta' \leftarrow \mathcal{A}_2^{\blue{\Odecrypt(\cdot)}}(pp, pk, state, c^*);
	\end{array} 
\right] - \frac{1}{2},
\end{displaymath}
在上述定义中, $\mathcal{A} = (\mathcal{A}_1, \mathcal{A}_2)$表示敌手$\mathcal{A}$可划分为两个阶段, 
划分界线是接收到挑战密文$c^*$前后, $state$表示$\mathcal{A}_1$向$\mathcal{A}_2$传递的信息, 记录部分攻击进展.  
$\Odecrypt(\cdot)$表示解密谕言机, 其在接收到密文$c$的询问后输出$\mathsf{Decrypt}(sk, c)$. 
如果任意的PPT敌手$\mathcal{A}$在上述游戏中的优势函数均为可忽略函数, 则称公钥加密方案是IND-CPA安全的;  
如果任意的PPT敌手在阶段1可自适应访问$\Odecrypt(\cdot)$的情形下仍仅具有可忽略优势, 则称公钥加密方案是IND-CCA1安全的; 
如果任意的PPT敌手在阶段1和阶段2均可自适应访问$\Odecrypt(\cdot)$的情形下仍仅具有可忽略优势, 
则称公钥加密方案是IND-CCA2或IND-CCA安全的.  
\end{trivlist}

以下阐述公钥加密安全性定义的一些细微之处: 
\begin{itemize}
\item 自适应的含义是敌手的攻击行为可根据学习到的知识动态调整, 如我们称敌手能够自适应的访问解密谕言机指敌手可以根据历史询问结果发起新的询问. 
	简而言之, 自适应性极大的增强了敌手的攻击能力.  

\item IND-CCA安全性远强于IND-CCA1和IND-CPA安全性, 这是因为敌手可以在观察到挑战密文$c^*$后有针对性的发起更加有威胁的解密询问. 

\item $(m_0, m_1)$由敌手任意选择, 从而巧妙精准的刻画了密文不泄漏明文任何一比特信息的直觉. 

\item 为了避免定义无意义, 在IND-CCA的安全游戏中禁止敌手在第二阶段向$\Odecrypt(\cdot)$询问挑战密文$c^*$.  
\end{itemize}

\begin{note}
对于密码方案, 给出恰当的安全性定义非常重要: 一方面安全性定义必须足够强以刻画现实中存在的攻击, 
另一方面安全性定义不能过强使得无法构造满足其的密码方案. 公钥加密的安全性定义是逐渐演化的. 

上世纪70年代, Diffie和Hellman提出了公钥加密的概念, 随后Riverst、Shamir和Adi构造出了首个公钥加密方案——RSA加密. 
在这一阶段, 公钥加密的安全性仅具备符合直觉的单向性, 即在平均意义下从密文中恢复出明文是计算困难的. 
到了上世纪80年代, 人们逐渐认识到单向性并不能满足应用需求, 这是因为对于单向安全的公钥加密方案, 
敌手有可能从密文恢复出明文的部分信息, 而在应用中, 由于数据来源的多样性和不确定性, 
明文的每一比特都可能包含关键的机密信息(比如股票交易指令中的``买''或``卖'').  

1982年, Goldwasser和Micali~\cite{GM-STOC-1982}指出单向安全的不足, 提出了语义安全性(semantic security). 
语义安全性的直观含义是密文对敌手求解明文没有帮助. 严格定义颇为精妙, 定义的形式是基于模拟的, 
即敌手掌握密文的视角可以由一个PPT的模拟器在计算意义下模拟出来. 
语义安全性可以看做Shannon完美安全性在计算意义的推广放松, 然而在论证的时候稍显笨重. 

Goldwasser和Micali给出了另一个等价的定义(等价性的证明参见Dodis和Ruhl的短文~\cite{DR-Web-1999}), 
即选择明文攻击下的不可区分性(IND-CPA, indistinguishability against chosen-plaintext attack). 
IND-CPA安全定义的直觉是密文在计算意义下不泄漏明文的任意一比特信息, 即对任意两个明文对应的密文分布是计算不可区分的, 
其中选择明文攻击刻画了公钥公开特性使得任意敌手均可通过自行加密获得任意明文对应密文这一事实. 
使用IND-CPA安全进行安全论证相比语义安全要便捷很多, 因此被广为采用. 

注意到IND-CPA安全仅考虑被动敌手, 即敌手只窃听信道上的密文. 
1990年, Naor和Yung~\cite{NY-STOC-1990}认为敌手有能力发起一系列主动攻击, 比如重放密文、修改密文等, 
进而提出选择密文攻击(CCA, chosen-ciphertext attack)刻画这一系列主动攻击行为, 即敌手可以自适应的获取指定密文对应的明文. 
Naor和Yung考虑了两种选择密文攻击, 一种是弱化版本, 
称为午餐时间攻击(lunch-time attack), 含义是敌手只能在极短的时间窗口(收到挑战密文之前)进行选择密文攻击; 
另一种是标准版本, 敌手可以长时间窗口(收到挑战密文前后)进行选择密文攻击. 

1998年, Bleichenbacher~\cite{Bleichenbacher-CRYPTO-1998}展示了针对PKCS\#1标准中公钥加密方案的有效选择密文攻击, 实证了关于选择密文安全的研究并非杞人忧天. 
Shoup~\cite{Shoup-TechReport-1998}进一步深入探讨了选择密文安全的重要性与必要性. 
从此, IND-CCA安全成为了公钥加密方案的事实标准.        
\end{note}


事实上, 公钥加密还存在另外一种更符合直觉的安全定义, 大意是对于明文空间中的随机明文进行加密, 
敌手正确猜测出加密明文的概率与随机猜测的正确概率相近. 我们称这种安全性为消息恢复选择明文安全(message-recovery-CPA security), 严格定义如下: 
\begin{trivlist}
\item \textbf{消息恢复选择明文安全.} 令$M$是明文空间. 定义公钥加密方案敌手$\mathcal{A}$的优势函数如下: 
\begin{displaymath}
    \AdvA(\lambda) =
    \Pr \left[ i = i':
    \begin{array}{l}
        pp \leftarrow \mathsf{Setup}(1^\kappa); \\
        (pk, sk) \leftarrow \mathsf{KeyGen}(pp);\\
        M' = \{m_1, \dots, m_n\} \subseteq M \leftarrow \mathcal{A}, |M'| \geq 2; \\ 
        i \sample [n];\\
        c^* \leftarrow \mathsf{Encrypt}(pk, m_i);\\
        i' \leftarrow \mathcal{A}(pk, M', c^*);
    \end{array} 
    \right] - \frac{1}{n}
\end{displaymath}
如果任意的PPT敌手$\mathcal{A}$在上述游戏中的优势函数均为可忽略函数, 则称公钥加密方案是Message-Recovery-CPA安全的.  
\end{trivlist}

\begin{theorem}
公钥加密的IND-CPA与Message-Recovery-CPA安全性是等价的.  
\end{theorem}
\begin{proof}
\begin{trivlist}
\item \ul{Message-Recovery-CPA $\Rightarrow$ IND-CPA.} 容易看出, 当限定$|M'| = 2$时, MessageRecovery-CPA蕴含IND-CPA.  
    令$\mathcal{A}$是针对IND-CPA安全的敌手, 以下展示如何基于$\mathcal{A}$构造$\mathcal{B}$打破MessageRecovery-CPA安全. 
    $\mathcal{B}$选择两个互异的明文$m_0, m_1$, 提交$M' = \{m_0, m_1\}$, 
    获得$c^* \leftarrow \mathsf{Encrypt}(pk, m_\beta)$作为挑战. 
    $\mathcal{B}$将$c^*$发送给$\mathcal{A}$, 并将$\mathcal{A}$的输出作为自己的输出发送给MessageRecovery-CPA的挑战者.  
    显然, $\mathcal{B}$完美的模拟了IND-CPA安全实验, $\mathsf{Adv}_\mathcal{B}(\kappa) = \AdvA$. 

\item \ul{IND-CPA $\Rightarrow$ Message-Recovery-CPA.} 我们通过以下的游戏序列完成证明: 

\item \text{Game 0.} 对应标准的MessageRecovery-CPA安全实验. 在挑战阶段, 
    $\mathcal{CH}$在收到敌手选择的$M'$后, 随机选择$i \sample [n]$, 
    发送$c^* \leftarrow \mathsf{Encrypt}(pk, m_i)$ to $\mathcal{A}$. 根据定义, 我们有: 
    \begin{equation*}
    	\AdvA(\kappa) = |\Pr[S_0] - 1/|M'||
    \end{equation*}

\item \text{Game 1.} 与标准的Message-Recovery-CPA安全实验基本相同, 唯一的不同是在挑战阶段收到$M'$后, 
	$\mathcal{CH}$随机选择$i \sample [n]$, $j \sample [n]$, 
    发送$c^* \leftarrow \mathsf{Encrypt}(pk, m_j)$ to $\mathcal{A}$. 
    因为$i$在信息论意义下隐藏于$\mathcal{A}$, 因此我们有: 
    \begin{equation*}
    	\Pr[S_1] = |1/M'|
    \end{equation*}
    此处需要特别强调上述概率仅建立在$\mathcal{CH}$随机选择$i$的随机带上. 

\item 最后, 我们证明IND-CPA安全保证了$|\Pr[S_0] - \Pr[S_1]| = \mathsf{negl}(\kappa)$. 
\item 令$\mathcal{B}$是针对IND-CPA安全的敌手. $\mathcal{B}$随机选择$i \sample [n], j \sample [n]$, 
    令$m_0 = m_i$, $m_1 = m_j$, 将$(m_0, m_1)$提交给它的挑战者并获得关于$m_\beta$的挑战密文$c^*$. 
    $\mathcal{B}$发送挑战密文$c^*$给$\mathcal{A}$, $\mathcal{A}$输出对$i$的猜测$i'$. 
    如果$i' = i$, $\mathcal{B}$输出$0$. 否则, $\mathcal{B}$输出对$\beta$的随机猜测. 
    当$\beta=0$时, $\mathcal{B}$向$\mathcal{A}$完美的模拟了$\text{Game}_0$, 
    因此正确输出$\beta$的概率是$\Pr[S_0]$; 
    当$\beta=1$时, $\mathcal{B}$向$\mathcal{A}$完美的模拟了$\text{Game}_1$, 
    因此正确输出$\beta$的概率是$(1-\Pr[S_1])$; 
    我们有: 
\begin{gather*}
    \mathsf{Adv}_\mathcal{B}(\kappa) = |\Pr[\beta = 0] \Pr[S_0] + \Pr[\beta = 1] (1-\Pr[S_1])-1/2| \\
    = \left|\frac{1}{2}(\Pr[S_0] - \Pr[S_1])\right|
\end{gather*}

\item IND-CPA的安全性保证了$\mathsf{Adv}_\mathcal{B}(\kappa)$可忽略, 因此$|\Pr[S_0] - \Pr[S_1]| \leq \mathsf{negl}(\kappa)$. 
综上我们有$\AdvA(\kappa) = \mathsf{negl}(\kappa)$. 
\end{trivlist}
以上展示了Message-Recovery-CPA和IND-CPA的双向蕴含, 定理得证. 
\end{proof}


\subsubsection{基本性质}
\begin{trivlist}
\item \textbf{同态性.} 公钥加密方案的正确性隐式保证了解密算法自然诱导出从密文空间$C$到明文空间$M$的一个映射$\phi = \mathsf{Dec}(sk, \cdot)$. 
    如果$\phi$具备同态性, 则第三方可对密文进行相应的公开计算, 得到的密文与对明文施加同样计算所得结果对应.  
    正式的, 令$\mathcal{C} = \{f\}$是从$M^n \rightarrow M$的某个电路族, 其中$n$是正整数; 
    $\mathsf{Eval}$为密文求值算法, 以公钥$pk$、$f \in \mathcal{C}$和密文向量$\mathbf{c} = (c_1, \dots, c_n)$为输入, 输出$c' \in C$, 
    记作$c' \leftarrow \mathsf{Eval}(pk, f, \mathbf{c})$. 
    如果对于任意$f \in \mathcal{C}$和任意明文$\mathbf{m} = (m_1, \dots, m_n) \in M^n$以下公式成立:
\begin{displaymath}
    \Pr \left[\mathsf{Dec}(sk, c') = f(\mathbf{m}):
    \begin{array}{l}
        (pk, sk) \leftarrow \mathsf{KeyGen}(1^\kappa);\\
        \mathbf{c} \leftarrow \mathsf{Enc}(pk, \mathbf{m});\\
        c' \leftarrow \mathsf{Eval}(pk, f, \mathbf{c});  
    \end{array} 
    \right] = 1
\end{displaymath}
则称公钥加密方案是$\mathcal{C}$-同态的, $\mathcal{C}$刻画了同态所支持的公开计算类型. 两种常见的同态类型如下: 
\begin{itemize}
\item 部分同态(partially homomorphic): 不失一般性, 若明文空间$M$为加法群, 密文空间$C$为乘法群, 若$\mathcal{C}$仅包含$M^2 \rightarrow M$的群运算, 
    则称加密方案是部分同态或者加法同态的. 此时同态性刻画如下: 
\begin{displaymath}
    \Pr \left[\mathsf{Dec}(sk, c_1 \cdot c_2) = f(m_1, m_2):
    \begin{array}{l}
        (pk, sk) \leftarrow \mathsf{KeyGen}(1^\kappa);\\
        c_1 \leftarrow \mathsf{Enc}(pk, m_1), c_2 \leftarrow \mathsf{Enc}(pk, m_2);\\
    \end{array} 
    \right] = 1
\end{displaymath} 

\item 全同态(fully homomorphic): 若$\mathcal{C}$包含了$M^n \rightarrow M$的所有多项式时间可计算函数, 则称方案是全同态的. 
\end{itemize}
\end{trivlist}

\begin{note}
几乎所有公钥加密方案都构建在代数性质良好的结构上, 且大部分方案均天然满足部分同态, 
如RSA、ElGamal、Goldwasser–Micali、Benaloh、Paillier、Sander-Young-Yung、Boneh–Goh–Nissim、Ishai-Paskin等. 
在RSA公钥加密方案横空出世仅一年后, Rivest、Adleman和Dertouzos~\cite{RAD-FOSC-1978}即提出了全同态公钥加密的概念. 
直到31年后, 才由Gentry~\cite{Gentry-STOC-2009}通过引入理想格构给出首个全同态加密方案的构造. 
自此突破之后, 全同态加密迅猛发展, 理论成果百花齐放, 效率不断提升, 成为了隐私保护技术中重要且实用的密码学工具. 
感兴趣的读者请参阅Halevi的综述文章~\cite{Halevi-Tutorial-2017}.
\end{note}

\begin{remark}
对于密码方案和协议, 安全性和功能效率之间通常存在权衡关系(trade-off). 
对于公钥加密方案, CPA安全性与同态性可以共存, 而更强的CCA安全性与同态性之间就存在冲突, 无法兼得. 
在现实世界中应用公钥加密方案时, 需根据应用场景的具体需求在安全性和功能效率之间做出恰当的选择, 切不可教条. 
\end{remark}


\begin{center}
\begin{tikzpicture}
    \node[circlenode, name=C, minimum width=6em, fill=gray!30] {$C$}; 
    \node[circlenode, name=M, minimum width=4em, right of = C, xshift=16em] {$M$};
    \draw (C) edge[->] node[above] {$\phi = \mathsf{Dec}(sk, \cdot)$} (M); 
\end{tikzpicture}
\end{center}


\subsection{密钥封装机制}\label{subsec:KEM}
主流的公钥加密方案基于数论或者格基困难问题构造. 
基于数论问题的公钥加密方案因需要进行高精度算术运算导致加解密速率较低, 基于格基困难问题的公钥加密方案存在公钥和密文尺寸较大的问题. 
而对称加密方案因其功能简单, 仅需异或等逻辑运算即可完成, 且硬件支持良好(如定制的指令), 
因此在较公钥加密具有较大的性能优势, 在加密长明文的场景下更为显著. 

如何解决公钥加密在加密长消息时的性能短板呢? 解决思路是混合加密(hybrid encryption), 
朴素的实现方式是PKE+SKE, 如图~\ref{figure:PKE+SKE}所示: 
\begin{enumerate}
\item 发送方首先随机选择对称密钥$k$, 调用公钥加密算法用接收方的公钥$pk$加密$k$得到$c$, 
	再调用对称加密算法用$k$加密明文$m$得到$c_1$, 最终的密文$(c, c')$. 

\item 接收方在接收到密文$(c, c')$后, 首先使用私钥$sk$解密$c$恢复对称密钥$k$, 再使用$k$解密$c'$. 
\end{enumerate}     

\begin{figure}[!hbtp]
\begin{center}
\begin{tikzpicture}
    \node [name=enc1, rectanglenode, minimum width=5em, minimum height=2em] {$\mathsf{PKE}(\cdot, \cdot)$}; 
    \node [name=k, textnode, above of=enc1, yshift=3em] {$k$}; 
    \node [name=c1, textnode, below of=enc1, yshift=-3em] {$c$}; 
    \node [name=pk, textnode, left of=enc1, xshift=-4em] {$pk$}; 
    \draw (pk) edge[->]  (enc1);
    \draw (k) edge[->] (enc1);
    \draw (enc1) edge[->] (c1);

    \node [name=enc2, rectanglenode, minimum width=5em, minimum height=2em, right of=enc1, xshift=10em] {$\mathsf{SKE}(\cdot, \cdot)$}; 
    \node [name=m, textnode, above of=enc2, yshift=3em] {$m$}; 
    \node [name=c2, textnode, below of=enc2, yshift=-3em] {$c'$}; 

    \draw (m) edge[connect] (enc2);
    \draw (enc2) edge[connect] (c2);

    \draw ($(enc1.north)+(0em, 1em)$) edge[-] ($(enc1.north)+(5em, 1em)$);
    \draw ($(enc1.north)+(5em, 1em)$) edge[-] ($(enc1.east)+(2.5em, 0em)$);
    \draw ($(enc1.east)+(2.5em, 0em)$) edge[->] (enc2);
\end{tikzpicture}
\end{center}
\caption{混合加密: PKE+SKE}\label{figure:PKE+SKE}
\end{figure}


混合加密方法既保留了公钥加密的功能性, 同时性能几乎与对称加密相当, 因此是公钥加密加密长明文时的通用范式. 
Cramer和Shoup~\cite{CS-EUROCRYPT-2002}观察到公钥加密在混合加密范式中起到的关键作用是发送方向接收方传输对称密钥, 
而传递的方式并非必须是加解密. 基于该观察, Cramer和Shoup提出了``密钥封装-数据封装''范式, 
简称为KEM-DEM(key/data-encapsulation mechanism), 该范式可以看作是混合加密的另一种实现方式, 如图~\ref{figure:KEM+DEM}所示.  
顾名思义, KEM-DEM范式包含KEM和DEM两个组件, DEM可以粗略的等同为对称加密, KEM是该范式的核心. 
简言之, KEM与PKE的不同在于发送方不再先显式选择对称密钥再加密, 而是封装一个随机的对称密钥. 

\begin{figure}[!hbtp]
\begin{center}
\begin{tikzpicture}
    \node [name=enc1, rectanglenode, minimum width=5em, minimum height=2em] {$\mathsf{KEM}(\cdot)$}; 
    \node [name=c1, textnode, below of=enc1, yshift=-3em] {$c$}; 
    \node [name=pk, textnode, left of=enc1, xshift=-4em] {$pk$}; 
    \draw (pk) edge[->]  (enc1);
    \draw (enc1) edge[->] (c1);

    \node [name=enc2, rectanglenode, xshift=10em, right of=enc1, minimum width=5em, minimum height=2em] {$\mathsf{DEM}(\cdot, \cdot)$}; 
    \node [name=m, textnode, above of=enc2, yshift=3em] {$m$}; 
    \node [name=c2, textnode, below of=enc2, yshift=-3em] {$c'$}; 

    \draw (m) edge[->] (enc2);
    \draw (enc2) edge[->] (c2);

    \draw (enc1) edge[->] node [auto] {$k$} (enc2);
\end{tikzpicture}
\end{center}
\caption{混合加密: KEM+DEM}\label{figure:KEM+DEM}
\end{figure}

\begin{definition}[密钥封装机制]
正式的, KEM由以下的四个多项式时间组成: 
\begin{itemize}
\item $\mathsf{Setup}(1^\kappa)$: 系统生成算法以安全参数$1^\lambda$为输入, 输出系统公开参数$pp$, 其中$pp$包含对
	公钥空间$PK$、私钥空间$SK$、对称密钥空间$K$和密文空间$C$的描述. 
	该算法由可信第三方生成并公开, 系统中的所有用户共享, 所有算法均将$pp$作为输入.  
	当上下文明确时, 常常为了行文简洁省去$pp$. 

\item $\mathsf{KeyGen}(pp)$: 密钥生成算法以系统公开参数$pp$为输入, 输出一对公/私钥对$(pk, sk)$, 其中公钥公开, 私钥秘密保存. 

\item $\mathsf{Encaps}(pk; r)$: 封装算法以公钥$pk \in PK$为输入, 输出对称密钥$k \in K$和封装密文$c \in C$.  

\item $\mathsf{Decaps}(sk, c)$: 解封装算法以私钥$sk \in SK$和密文$c \in C$为输入, 
	输出对称密钥$k \in K$或者$\bot$表示封装密文非法. 解封装算法通常为确定性算法. 
\end{itemize}
\end{definition}

\begin{remark}
在KEM中, 对称密钥$k$起到的作用是在发送方和接收方之间建立安全的会话信道, 因此也常称为会话密钥. 
\end{remark}

\begin{trivlist}
\item \textbf{正确性.} 该性质保证KEM的功能性, 即使用私钥可以正确恢复出封装密文所封装的会话密钥. 
正式的, 对于任意会话密钥$k \in K$, 有:
\begin{equation}\label{equation:KEM-correctness}
	\Pr[\mathsf{Decaps}(sk, c) = k: (c, k) \leftarrow \mathsf{Encaps}(pk)] = 1 - \mathsf{negl}(\kappa).
\end{equation}
公式~\eqref{equation:KEM-correctness}的概率建立在$\mathsf{Setup}(1^\kappa) \rightarrow pp$、
$\mathsf{KeyGen}(pp) \rightarrow (pk, sk)$和$\mathsf{Encaps}(pk) \rightarrow (c, k)$的随机带上. 
如果上述概率严格等于1, 则称KEM方案满足完美正确性. 
\end{trivlist}

\begin{trivlist}
\item \textbf{安全性.} 定义KEM敌手$\mathcal{A}$的优势函数如下: 
\begin{displaymath}
\AdvA(\kappa) = \Pr \left[ \beta' = \beta:~
	\begin{array}{ll}
		& pp \leftarrow \mathsf{Setup}(1^\lambda); \\		
		& (pk, sk) \leftarrow \mathsf{KeyGen}(pp);\\
        & (c^*, k_0^*) \leftarrow \mathsf{Encaps}(pk), k_1^* \sample K;\\
		& \beta \sample \{0,1\};\\ 
        & \beta' \leftarrow \mathcal{A}^{\blue{\Odecaps(\cdot)}}(pp, pk, c^*, k_\beta^*);
	\end{array} 
\right] - \frac{1}{2},
\end{displaymath}
在上述定义中, $\Odecaps(\cdot)$表示解封装谕言机, 其在接收到密文$c$的询问后输出$\mathsf{Decaps}(sk, c)$. 
如果任意的PPT敌手$\mathcal{A}$在上述游戏中的优势函数均为可忽略函数, 则称KEM方案是IND-CPA安全的;   
如果任意的PPT敌手在可自适应访问$\Odecaps(\cdot)$的情形下仍仅具有可忽略优势, 则称KEM方案是IND-CCA安全的. 
\end{trivlist}

下面给出安全性定义的一些注记: 
\begin{itemize}
\item KEM的安全游戏中分阶段定义敌手不再必要, 因为挑战密文的生成不受敌手控制, 
	正是这点不同使得KEM的安全性定义要比PKE的安全性定义简单. 

\item 为了避免定义无意义, 在IND-CCA的安全游戏中禁止敌手向$\Odecaps(\cdot)$询问挑战密文$c^*$.  
\end{itemize}

\magenta{补充DEM部分和KEM+DEM的安全性组合}

\subsection{两类混合加密范式的比较}
以上两种混合方法的共性都是首先生成对称密钥, 再利用对称密钥加密明文, 因此效率方面的差异体现在第一阶段. 
PKE+SKE范式的非对称部分是先选择一个随机的密钥$k$, 再使用PKE对其加密得到$c$, 而KEM+DEM范式的非对称部分是两步并做一步完成. 
如果使用PKE-SKE范式, 密文$c$必然存在密文扩张, 这是由概率加密的本质决定的; 
而如果使用KEM+DEM的方法, 密文$c$相比$k$可能不存在扩张, 原因是此时$c$是对$k$的封装, 而非加密. 
综上, 使用KEM代替PKE, 不仅能够缩减整体密文尺寸, 也能够提升效率. 
\begin{remark}
通常KEM要比PKE构造简单, 这是因为KEM可以看作功能受限的PKE, 因为其只允许加密随机的明文.
\end{remark}

相比效率提升, KEM-DEM的理论价值更大. 首先, KEM-DEM范式实现了对PKE的功能解耦, 将PKE中的非对称内核抽取出来凝练为KEM, 意义如下:
\begin{itemize}
\item KEM-DEM范式极大简化了PKE的可证明安全. 我们只需证明KEM和DEM满足一定性质即可. 
	对比安全模型即可发现, 对于PKE有CPA/CCA1/CCA三个依次增强的安全性, 而KEM只有CPA/CCA两个依次增强的安全性. 
	最关键的是: 在PKE中敌手A对挑战密文$c^*$有一定的控制能力, 而KEM中$c^*$完全由挑战者控制, 
	这一区别使得KEM安全证明中的归约算法更容易设计.

\item KEM-DEM范式有助于简化PKE的设计. 该范式将PKE的设计任务简化为对应的KEM, 
	在后面的章节中可以看到, 在设计高等级安全的PKE时, 仅需设计满足相应安全性的KEM即可.

\item KEM-DEM范式有助于洞悉PKE本质. 该范式揭示了构造PKE的核心机制在于构造KEM. 
	后续的章节揭示了KEM的本质是公开可求值的伪随机函数, 是伪随机函数在minicrypt中的对应. 
	认识到这一点后, 不仅可以将几乎所有公钥加密的构造统一在同一框架下, 还可以将SKE和PKE的构造在伪随机函数的视角下实现高度统一. 
\end{itemize}

\begin{remark}
目前, 格基的KEM设计仍是PKE-SKE的方式, 显得不够灵巧纯粹, 如何设计精巧纯粹的格基KEM是很有挑战意义的研究课题. 
\end{remark}