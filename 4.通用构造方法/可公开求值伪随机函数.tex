\section{可公开求值伪随机函数类}
\begin{center}
    明修栈道, 暗渡陈仓.  \\
                \hfill --- 汉 $\cdot$ 司马迁《史记·高祖本纪》
\end{center}

前面的章节已经展示了若干种构造公钥加密的通用方法, 包括单向陷门函数、哈希证明系统、可提取哈希证明系统以及不可区分混淆结合可穿孔伪随机函数. 
这些通用构造阐释了绝大多数公钥加密方案, 然而令人颇感以外的是, 它们并无法阐释最经典的ElGamal PKE~\cite{ElGamal-IEEE-IT-1985}
和Goldwasser-Micali PKE~\cite{GM-JCSS-1984}
另一方面, 伪随机函数是密码学的核心基本组件之一, 应用范围极其广泛, 
特别的, 伪随机函数蕴含了简洁优雅的、也是目前唯一的IND-CPA SKE通用构造.
\begin{equation*}
    \mathsf{Enc}(sk, m; r) \rightarrow (r, F(sk, x) \oplus m)
\end{equation*}
然而伪随机函数属于$\mathsf{Minicrypt}$, 因此在黑盒意义下无法蕴含PKE. 

以上的现象促使我们考虑如下的问题: 
\begin{thinking}
    是否存在新型的伪随机函数能够让PRF-based SKE延拓到公钥场景? 新型的伪随机函数是否能蕴含统一上述的不同构造, 并阐释经典PKE方案的设计机理?
\end{thinking}

我们首先分析基于PRF构造PKE的技术难点: 
\begin{itemize}
\item 密文必须可以公开计算: 显然, PRF-based SKE的构造正是因为这个原因无法延拓到公钥加密场景中
\begin{equation*}
    (x, F(sk, x) \oplus m)
\end{equation*}
因为$F$的伪随机性意味着其不可能公开求值.  
\end{itemize}

解决上述问题的关键在于探求伪随机性(pseudorandomness)和可公开求值性(public evaluability)是否能够共存.  
标准的伪随机函数处处伪随机(universal pseudorandom), 即对于定义域中任意$x \in X$, PPT敌手$\mathcal{A}$都无法区分$F_k(x)$和随机值. 
\begin{itemize}
\item 观察1: 构造IND-CPA KEM仅需要弱伪随机性(weak pseudorandomness), 即对于挑战者随机选择的挑战输入, 其PRF值是伪随机的 

\item 观察2: 如果掌握输入$x$的某些辅助信息$aux$ (比如采样$x$的随机数), 是有可能在不使用$sk$的情形下对$F_{sk}(x)$公开求值. 
	如果$aux$在平均意义下是难以抽取的, 则公开求值性与弱伪随机性不冲突.  
\end{itemize}

综合以上, 在KEM中由发送方生成$x$, 因此其知晓$aux$信息, 从而以下两点成为可能: 
\begin{itemize}
    \item 功能性方面: 发送方可以借助$aux$对$F_{sk}(x)$公开求值从而生成密文.  
    \item 安全性方面: $F_{sk}(x)$在$\mathcal{A}$的视图中仍然伪随机.  
\end{itemize}

\subsection{定义与安全性}
正是基于上面的思考, 陈等~\cite{Chen-SCN-2014}提出了可公开求值伪随机函数(PEPRF, Publicly Evaluable PRFs). 
PEPRF考虑了定义域$X$包含$\mathcal{NP}$语言$L$的情形, 使用私钥可以对全域求值, 而使用公钥和证据可以对语言$L$内的元素求值. 
在安全性上, PEPRF要求函数在语言$L$上弱伪随机.  
\begin{definition}[可公开求值伪随机函数]\label{definition:PEPRF}
\begin{itemize} 
\item $\mathsf{Setup}(1^\kappa)$: 以安全参数$\kappa$为输入, 输出公开参数$pp = (F, PK, SK, X, L, W, Y)$, 
    其中$F: SK \times X \rightarrow Y \cup \bot$是由$SK$索引的一族函数, 
    $L \subseteq X$是由困难关系$\mathsf{R}_L$定义的$\mathcal{NP}$语言, 其中$W$是相应的证据集合. 
    $\mathsf{R}_L$是高效可采样的, 存在PPT算法$\mathsf{SampR}$以随机数$r$为输入, 
    输出实例证据元组$(x, w) \in \mathsf{R}_L$.   

\item $\mathsf{KeyGen}(pp)$: 以公开参数$pp$为输入, 输出公钥$pk$和私钥$sk$.   
    
\item $\mathsf{PrivEval}(sk, x)$: 以私钥$sk$和元素$x \in X$为输入, 输出$y \in Y \cup \bot$.  

\item $\mathsf{PubEval}(pk, x, w)$: 以公钥$pk$、实例$x \in L$以及相应的证据$w \in W$为输入, 输出$y \in Y$.  
\end{itemize}
\end{definition}

\begin{remark} 
在有些场景中有必要将单一语言$L$泛化为由$PK$索引的一族语言$\{L_{pk}\}_{pk \in PK}$. 
相应的, 采样算法$\mathsf{SampRel}$将以$pk$为额外输入, 随机采样$(x, w) \in \mathsf{R}_{L_{pk}}$. 
\end{remark}

\begin{trivlist}
\item \textbf{正确性.} 对于任意$pp \leftarrow \mathsf{Setup}(1^\lambda)$和$(pk, sk) \leftarrow \mathsf{KeyGen}(pp)$, 我们有:
\begin{eqnarray*}
    & \forall x \in X: & F_{sk}(x) = \mathsf{PrivEval}(sk, x)\\
    & \forall x \in L \text{~以及证据~} w: & F_{sk}(x) = \mathsf{PubEval}(pk, x, w)
\end{eqnarray*} 
\end{trivlist}

\begin{trivlist}
\item \textbf{(自适应)弱伪随机性.} 定义敌手$\mathcal{A}$的优势函数如下: 

\begin{displaymath}
\Pr \left[\beta = \beta':   
    \begin{array}{l}
        pp \leftarrow \mathsf{Setup}(1^\kappa); \\
        (pk, sk) \leftarrow \mathsf{KeyGen}(pp);\\
        r^* \sample R, (x^*, w^*) \leftarrow \mathsf{SampR}(r^*);\\
        y_0^* \leftarrow F_{sk}(x^*), y_1^* \leftarrow Y;\\
        \beta \leftarrow \{0,1\};\\     
        \beta' \leftarrow \mathcal{A}^{\red{\Oeval(\cdot)}}(pp, pk, x^*, y_b^*);
    \end{array} 
\right] - \frac{1}{2}. 
\end{displaymath}
其中$\Oeval$表示求值谕言机, 以$x \neq x^* \in X$为输入, 返回$F_{sk}(x)$.  
如果任意PPT敌手$\mathcal{A}$在上述游戏中的优势函数均为可忽略函数, 则称可公开求值伪随机函数是弱伪随机的. 
如果敌手在上述游戏中可以访问$\Oeval$谕言机, 则称可公开求值伪随机函数是自适应弱伪随机的. 
\end{trivlist}

\begin{figure}[!hbtp]
\begin{center}
\begin{tikzpicture}
\node [textnode, name = Setup] {$\mathsf{Setup}(1^\kappa) \rightarrow pp$}; 
\node [textnode, name = KeyGen, below of = Setup, yshift=-2em] {$\mathsf{KeyGen}(pp) \rightarrow (pk, sk)$}; 

\node [circlenode, name=domain, draw=gray!50, fill=gray!50, minimum size=6em, below of = Setup, 
    xshift=-8em, yshift=-9em, label={[xshift=-1.5em, yshift=-2em]$X$}] {}; 
\node [name=language, circlenode, draw=violet!70, fill=violet!70, minimum size=2.5em, below of = domain, 
    yshift=-1em, xshift=0.5em] {$L$};

\node [name=witness, circlenode, draw=cyan, fill=cyan, minimum size=2.5em, below of = language, yshift=-5em] {$W$};

\node [name=sample, textnode, below of = language, xshift=-4em, yshift=-2.5em] {$\mathsf{SampR}(r)$};

\draw (language) edge[<->] node[auto] {$\mathsf{R}_L$} (witness);
\draw ($(sample.east)$) edge[->] (language);
\draw ($(sample.east)$) edge[->] (witness); 

\node [circlenode, name=range, draw=blue!20, fill=blue!20, minimum size=4em, right of = domain, xshift=16em] {$Y$}; 
\draw (domain) edge[->] node[auto] {$F(sk, x)$} (range);

\node [name=x1, circlenode, draw=black, fill=black, minimum size=0.1em, inner sep = 0, 
    above of = language, xshift=1em, yshift=2.2em] {};
\node [name=y1, circlenode, draw=black, fill=black, minimum size=0.1em, inner sep = 0, 
    left of = range, xshift=-1em, yshift=0.75em] {}; 
\draw (x1) edge[->, color=red, bend left=35] node[auto, xshift=0em] {$\mathsf{PrivEval}(sk, x)$} (y1);

\node [name=x2, circlenode, draw=black, fill=black, right of = language, 
    minimum size=0.1em, inner sep = 0, xshift=0.5em, yshift=0.2em] {};
\node [name=y2, circlenode, draw=black, fill=black, left of = range, 
    minimum size=0.1em, inner sep = 0, xshift=-1em, yshift=-0.75em] {}; 
\draw (x2) edge[->, color=blue, bend left=-35] node[auto, swap, xshift=1em] {$\mathsf{PubEval}(pk, x, w)$} (y2);
\end{tikzpicture}
\end{center}
\caption{PEPRF示意图}
\end{figure}

\begin{remark}
在PEPRF中, 私钥用于秘密求值, 公钥则可在知晓相应证据时对语言内的元素进行公开求值. 
密钥成对出现这一点对于PEPRF是自然的, 因为PEPRF是作为PRF在$\mathsf{Cryptomania}$中的对应引入的. 
另一方面, 标准的PRF也总是可以设置公钥用于发布与私钥相关联但可公开的信息, 例如在基于DDH假设的Naor-Reingold PRF~\cite{NR-JACM-2004}中, 
$F_{\vec{a}}(x) = (g^{a_0})^{\prod_{x_i=1}a_i}$, 
其中$\vec{a} = (a_0, a_1, \dots, a_n) \in \mathbb{Z}_p^n$是私钥, 
$\{g^{a_i}\}_{1 \leq i \leq n}$则可发布为公钥.  
如果没有信息可公开, 可设定$pk = \{\bot\}$. 如此可保持PRF与PEPRF的语法定义保持一致.    

为什么PEPRF只定义了弱伪随机性呢? 这是因为在公开求值算法$\mathsf{PubEval}$存在的前提下, 这是可达的最强安全性.  
\end{remark}

为了加深对概念的理解, 表~\ref{table:PRF-vs-PEPRF}对比分析PRF与PEPRF的异同. 
\begin{table}
\centering
\begin{tabular}{|c|c|c|}
\hline
               & PRF & PEPRF \\
\hline
带密钥函数     &  \checkmark  & \checkmark     \\
\hline
可公开求值 & $\forall x \in X$ \ding{55}  &  $x \in L$ $\checkmark$     \\
\hline
安全性       & $\forall x \in X$  pseudorandom & $x \sample L$ weak pseudorandom\\
\hline
\end{tabular}
\caption{PRF与PEPRF的比较}\label{table:PRF-vs-PEPRF}
\end{table}
正是由于上述区别, 我们可以基于PEPRF构造KEM. 

PEPRF的定义可以进一步泛化以包容更多实例化构造. 
\begin{definition}[可公开采样伪随机函数(PSPRF, Publicly Sampleable PRF)]
PSPRF将PEPRF的可公开求值功能可以放宽为可公开采样功能, 即$\mathsf{PubEval}$算法由以下的PPT随机采样算法替代: 
\begin{itemize}
    \item $\mathsf{PubSamp}(pk; r) \rightarrow (x, y) \in L \times Y \text{~s.t.~} y = F_{sk}(x)$
\end{itemize}
\end{definition}

\begin{figure}[!hbtp]
\begin{center}
\begin{tikzpicture}
\node [name=PE, textnode] {$F_{sk}(x)$可公开求值}; 
\node [name=PS, textnode, below of = PE, yshift=-5em] {$(x, F_{sk}(x))$可公开采样};

\draw (PE) edge[->, thick, dotted] (PS);

\node [name=domain, circlenode, draw=none, fill=gray!50, minimum size=5em, right of = PE, xshift=8em, yshift=-2.5em] {};
\node [textnode, above of = domain, yshift=1.5em, xshift=-1.3em] {$X$}; 
\node [name=language, circlenode, draw=none, fill=violet!70, minimum size=2em, right of = domain, xshift=1em] {$L$};
\node [name=range, circlenode, draw=none, fill=blue!20, minimum size=4em,  right of = domain, xshift=12em] {$Y$}; 
\node [name=x1, dotnode, draw=black, fill=black, right of = language, xshift=0.5em, yshift=0.5em] {};
\node [name=x2, dotnode, draw=black, fill=black, right of = language, xshift=0.5em, yshift=-0.5em] {};
\node [name=y1, dotnode, draw=black, fill=black, left of = range, xshift=-1em, yshift=0.6em] {}; 
\node [name=y2, dotnode, draw=black, fill=black, left of = range, xshift=-1em, yshift=-0.6em] {};

\draw (x1) edge[->, color=blue, bend left=30] node[above] {$\mathsf{PubEval}(pk, x, w)$} (y1);
\draw (x2) edge[<->, dashed, color=darkgreen, bend left=-30] node[below] {$\mathsf{PubSamp}(pk; r)$} (y2);

\end{tikzpicture}
\end{center}
\end{figure}

显然, 可以综合关系采样算法和函数公开求值算法构造公开采样算法, 因此PEPRF蕴含PSPRF: 
\begin{itemize}
    \item $\mathsf{PubSamp}(pk; r)$: 运行$(x, w) \leftarrow \mathsf{SampRel}(r)$, 
        输出$(x, \text{PEPRF}.\mathsf{PubEval}(pk, x, w))$. 
\end{itemize}

\subsection{基于PEPRF的KEM构造}
本章将展示如何基于PEPRF构造KEM. 

\begin{construction}[基于PEPRF的KEM构造]\label{construction:KEM-from-PEPRF}
\begin{trivlist}
    \item 构造思路: 随机采样语言中的元素作为密文, 计算其函数值作为会话密钥$k$. 

    \item 起点: PEPRF $F: SK \times X \rightarrow Y \cup \bot$, 其中$L \subseteq X$是定义在$X$上的$\mathcal{NP}$语言. 
\end{trivlist}
构造如下: 
\begin{itemize}
    \item $\mathsf{Setup}(1^\kappa)$: 运行$pp \leftarrow \text{PEPRF}.\mathsf{Setup}(1^\kappa)$, 
        其中密文空间$C = X$, 会话密钥空间$K = Y$. 

    \item $\mathsf{KeyGen}(pp)$: 运行$(pk, sk) \leftarrow \text{PEPRF}.\mathsf{KeyGen}(pp)$. 

    \item $\mathsf{Encaps}(pk; r)$: 随机采样$(x, w) \leftarrow \mathsf{SampR}(r)$, 输出$c = x$作为密文, 
        通过$\text{PEPRF}.\mathsf{PubEval}(pk, x, w)$公开计算$k \leftarrow F_{sk}(x)$作为会话密钥. 

    \item $\mathsf{Decaps}(sk, c)$: 通过运行$\text{PEPRF}.\mathsf{PrivEval}(sk, c)$秘密计算$k \leftarrow F_{sk}(x)$恢复会话密钥.
\end{itemize}
\end{construction}

构造~\ref{construction:KEM-from-PEPRF}的正确性由PEPRF的正确性保证, 安全性由以下定理保证. 

\begin{theorem}
如果PEPRF是弱伪随机的, 则构造~\ref{construction:KEM-from-PEPRF}是IND-CPA安全的; 
如果PEPRF是自适应弱伪随机的, 则构造~\ref{construction:KEM-from-PEPRF}是IND-CCA安全的.  
\end{theorem}

\begin{proof}
IND-CPA安全性的归约是显然的, 建立IND-CCA安全性的关键是令归约算法利用$\Oeval$模拟$\Odecaps$. \qed
\end{proof}

\begin{remark}
在上述的KEM构造中, 可以将PEPRF弱化为PSPRF. 
\end{remark}

\subsection{PEPRF的构造}
\begin{center}
    天下同归而殊途, 一致而百虑.  \\
                \hfill --- 《周易·系辞下》
\end{center}

本章节展示如何基于具体的困难假设和(半)通用的密码组件构造PEPRF.

\subsubsection{基于DDH假设的PEPRF}
图~\ref{figure:DDH-based-PEPRF}展示了基于DDH假设的PEPRF构造, 
其中可公开求值功能利用了DH函数的可交换性, 弱伪随机性建立在DDH假设之上. 
将实例化代入构造~\ref{construction:KEM-from-PEPRF}中, 得到的正是经典的ElGamal PKE方案~\cite{ElGamal-IEEE-IT-1985}. 
\begin{construction}[基于DDH假设的PEPRF]\label{construction:DDH-based-PEPRF}
\begin{itemize}
    \item $\mathsf{Setup}(1^\kappa)$: 运行$(\mathbb{G}, p, g) \leftarrow \mathsf{GroupGen}(1^\kappa)$, 
        生成公开参数$pp = (F, PK, SK, X, L, W, Y)$, 其中$X = Y = PK = L = \mathbb{G}$, $SK = W = \mathbb{Z}_p$, 
        $F: SK \times X \rightarrow Y$定义为$F_{sk}(x) = x^{sk}$, 
        语言$L = \{x: \exists w \in W \text{~s.t.~} x = g^w\}$, 
        相应的采样算法$\mathsf{SampRel}$以随机数$r$为输入, 随机采样证据$w \sample \mathbb{Z}_p$, 计算实例$x = g^w$. 

    \item $\mathsf{KeyGen}(pp)$: 随机采样私钥$sk \sample \mathbb{Z}_p$, 计算公钥$pk = g^{sk}$. 

    \item $\mathsf{PrivEval}(sk, x)$: 输出$x^{sk}$. 

    \item $\mathsf{PubEval}(pk, x, w)$: 输出$pk^w$. 
\end{itemize} 
\end{construction}

\begin{figure}[!hbtp]
\begin{center}
\begin{tikzpicture}
\node [name=pp, textnode] {$pp = (\mathbb{G}, p, g) \leftarrow \mathsf{Setup}(1^\kappa)$};
\node [name=Gen, textnode, below of=pp, yshift=-1.5em] {$\mathsf{KeyGen}(pp): sk \sample \mathbb{Z}_p, pk \leftarrow g^{sk}$};

\node [name=domain, circlenode, draw=violet!70, fill=violet!70, minimum size=3.5em, 
    below of = Gen, yshift=-5em, xshift=6em] {$X=L$};
\node [textnode, left of=domain, xshift=-3em] {$\mathbb{G}$};  

\node [name=witness, circlenode, draw=cyan, fill=cyan, minimum size=3.5em, 
    below of = domain, yshift=-7em] {$W$};
\node [textnode, left of=witness, xshift=-3em] {$\mathbb{Z}_p$};

\node [name=sample, textnode, below of = domain, xshift=-6em, yshift=-3.5em] {$\mathsf{SampR}(r)$}; 

\draw (domain) edge[<->] node[auto, name=relation] {$\mathsf{R}_L$} (witness);
\draw ($(sample.east)$) edge[->, thin] node[above] {$g^w$} (domain);
\draw ($(sample.east)$) edge[->, thin] node[below] {$w$} (witness); 

\node [name=range, circlenode, draw=blue!20, fill=blue!20, minimum size=3.5em, right of = domain, xshift=14em] {$Y$};

\node [textnode, name=description, below of=relation, xshift=8em, yshift=-3em] {\underline{$\{(x, w): g^w = x\}$}};
\draw (relation) edge[->, bend left=-25] (description.west);

\draw (domain) edge[->] node[auto] {$F(sk, x) = x^{sk}$} (range);

\node [name=x1, circlenode, draw=black, fill=black, above of = domain, 
    minimum size=0.1em, inner sep=0em, xshift=1.2em, yshift=1em] {};
\node [name=y1, circlenode, draw=black, fill=black, above of = range, 
    minimum size=0.1em, inner sep=0em, xshift=-1.2em, yshift=1em] {}; 
\draw (x1) edge[->, color=red, bend left=30] node[auto] {$\mathsf{PrivEval}(sk, x) = x^{sk}$} (y1);

\node [name=x2, circlenode, draw=black, fill=black, below of = domain, 
    minimum size=0.1em, inner sep = 0em, xshift=1.2em, yshift=-1em] {};
\node [name=y2, circlenode, draw=black, fill=black, below of = range, 
    minimum size=0.1em, inner sep = 0em, xshift=-1.2em, yshift=-1em] {}; 
\draw (x2) edge[->, color=blue, bend left=-25] node[auto, swap, xshift=1em] {$\mathsf{PubEval}(pk, x, w) = pk^w$} (y2);
\end{tikzpicture}
\end{center}
\caption{基于DDH假设的PEPRF}\label{figure:DDH-based-PEPRF}
\end{figure}

\subsubsection{基于QR假设的PEPRF}
图~\ref{figure:QR-based-PEPRF}展示了基于QR假设的PEPRF构造, 其中可公开求值功能利用了语言$L$的OR型定义, 弱伪随机性建立在QR假设之上. 
将实例化代入构造~\ref{construction:KEM-from-PEPRF}中, 得到的正是Goldwasser-Micali PKE方案~\cite{GM-JCSS-1984}内蕴的KEM. 
\begin{construction}[QR-based-PEPRF] 
\begin{itemize}
    \item $\mathsf{Setup}(1^\kappa)$: 输出$pp = \kappa$. 

    \item $\mathsf{KeyGen}(pp)$: 运行$(N, p, q) \leftarrow \mathsf{GenModulus}(1^\kappa)$, 
        选取$z \in \mathbb{QNR}_N^{+1}$, 输出公钥$pk=(N, z)$和私钥$sk = (p, q)$. 
        $pk$还包含了以下信息: 函数定义域$X = \mathbb{Z}_N^*$, 值域$Y = \{0,1\}$,  证据集合$W = \mathbb{Z}_N^*$, 
        语言$L_{pk} = \{x: \exists w \in W \text{~s.t.~} x = w^2 \bmod N \vee x = zw^2 \bmod N\}$($\mathbb{Z}_N^*$中Jacobi符号为$+1$的元素). 
        采样算法$\mathsf{SampRel}$以公钥$pk$和随机数$r$为输入, 随机采样$w \sample \mathbb{Z}_p$, 
        随机生成实例$x = w^2 \bmod N$或$x = z w^2 \bmod N$.  
    
    \item $\mathsf{PrivEval}(sk, x)$: 如果$x \in \mathbb{QR}_N$则输出$1$, 如果$x \in \mathbb{QNR}_N^{+1}$则输出$0$. 

    \item $\mathsf{PubEval}(pk, x, w)$: 如果$x = w^2 \bmod N$则输出$1$, 如果$x = z w^2 \bmod N$则输出$0$. 
\end{itemize} 
\end{construction}

\begin{figure}[!hbtp]
\begin{center}
\begin{tikzpicture}
\node [name=domain, circlenode, draw=black, minimum size=6em] {};
\fill[violet!70] (0, 0) -- (0, 3em) arc (90:270:3em) -- (0,0em);
\node [name=Gen, textnode, above of = domain, yshift=8em] {$pp=\kappa \leftarrow \mathsf{Setup}(1^\kappa)$};
\node [name=Gen, textnode, above of = domain, yshift=7em] 
    {$\mathsf{KeyGen}(pp): sk \leftarrow (p,q), pk \leftarrow (N, z \sample \mathbb{QNR}_N^{+1})$};

\draw (0,0) edge[color=black,-] (-3em,0);
\draw (-1.3em,1.1em) node {\scriptsize{$\mathbb{QR}_N$}};
\draw (-1.3em,-1.1em) node {\scriptsize{$\mathbb{QNR}_N^{+1}$}};
\draw (1.4em,0em) node {\scriptsize{$\mathbb{QNR}_N^{-1}$}};
\draw (-4em, 0em) node {$\mathbb{Z}_N^*$}; 
\draw (-6em,2em) node {$X=L=\mathbb{J}_N^{+1}$};
\draw (-1em, 3.7em) node {$\mathbb{J}_N^{+1}$}; 
\draw (1em, 3.7em) node {$\mathbb{J}_N^{-1}$}; 

\node [name=witness, circlenode, draw=cyan, fill=cyan, minimum size=3em, below of = domain, yshift=-8em, xshift=-1em] {$W$};
\node [textnode, left of=witness, xshift=-2.5em] {$\mathbb{Z}_N^*$};

\node [name=sample, textnode, below of = domain, xshift=-7em, yshift=-5em] {$\mathsf{SampRel}(r)$}; 

\draw ($(sample.east)$) edge[->] node[above, xshift=-2em] {$w^2$ or $zw^2$} (domain);
\draw ($(sample.east)$) edge[->] node[below, xshift=0em, yshift=0em] {$w$} (witness); 

\draw ($(domain.south)+(-1em, 0.2em)$) edge[<->] node[auto] {$\mathsf{R}_L$} (witness);

\node [name=R, textnode, right of=witness, xshift=11em] {\underline{$\{(x, w): x=w^2 \vee x=zw^2\}$}};

\draw (0.2em, -5.5em) edge[->, bend left=-25] (3em, -8em);

\node [name=range, circlenode, draw=blue!20, fill=blue!20, minimum size=3em, right of = domain, xshift=18em] {$Y$};
\draw (domain) edge[->] node[auto] {$F(sk, x) = \mathbb{J}_p(x) \wedge \mathbb{J}_q(x)$} (range);

\node [name=x1, circlenode, draw=black, fill=black, left of = domain, minimum size=0.1em, inner sep = 0em, 
    xshift=-1em, yshift=2em] {};
\node [name=y1, circlenode, draw=black, fill=black, left of = range, minimum size=0.1em, inner sep = 0em, 
    xshift=0em, yshift=1em] {}; 
\draw (x1) edge[->, color=red, bend left=20] node[auto, xshift=-6em] 
    {$\mathsf{PrivEval}(sk, x) = \mathbb{J}_p(x) \wedge \mathbb{J}_q(x)$} (y1);

\node [name=x2, circlenode, draw=black, fill=black, left of = domain, minimum size=0.1em, inner sep = 0em, 
    xshift=-1em, yshift=-2em] {};
\node [name=y2, circlenode, draw=black, fill=black, left of = range, minimum size=0.1em, inner sep = 0em, 
    xshift=0em, yshift=-1em] {}; 
\draw (x2) edge[->, color=blue, bend left=-15] node[auto, swap, xshift=-6em] 
    {$\mathsf{PubEval}(pk, x, w)= (x = w^2)?1,0$} (y2);
\end{tikzpicture}
\end{center}
\caption{基于QR假设的PEPRF构造}\label{figure:QR-based-PEPRF}
\end{figure}


\subsubsection{基于TDF的PEPRF}
通过扭转单射TDF, 可以构造PEPRF如下. 
\begin{construction}[基于TDF的PEPRF构造]\label{construction:TDF-based-PEPRF}
\begin{itemize}
    \item $\mathsf{Setup}(1^\kappa)$: 运行$pp = (G, EK, TD, S, U) \leftarrow \text{TDF}.\mathsf{Setup}(\lambda)$, 
        令$\mathsf{hc}: S \rightarrow K$是相应的hardcore function; 
        生成PEPRF的公开参数$pp = (F, PK, SK, X, L, W, Y)$, 其中$PK = EK$, $SK = TD$, $Y = K$, $X = U$, $W = S$, 
        $F_{sk}(x) = \mathsf{hc}(G_{td}^{-1}(x))$. 
        算法$\text{TDF}.\mathsf{Eval}$自然定义了一族定义在$X$上的$\mathcal{NP}$语言$L = \{L_{pk}\}_{pk \in PK}$, 
        其中$L_{pk} = \{x: \exists w \in W \text{~s.t.~} x = \text{TDF}.\mathsf{Eval}(pk, w)\}$.  
        采样算法$\mathsf{SampRel}$以随机数$r$为输入, 首先随机采样定义域中元素$s \leftarrow \mathsf{SampDom}(r)$, 
        再计算$u \leftarrow \text{TDF}.\mathsf{Eval}(pk, s)$, 输出实例$x = u$和证据$w = s$.   

    \item $\mathsf{KeyGen}(pp)$: 运行$(ek, td) \leftarrow \text{TDF}.\mathsf{KeyGen}(pp)$, 输出$pk = ek$和$sk = td$.  

    \item $\mathsf{PrivEval}(sk, x)$: 以私钥$sk$和元素$x \in X$为输入, 
        输出$y \leftarrow F_{sk}(x) = \mathsf{hc}(\text{TDF}.\mathsf{TdInv}(sk, x))$. .  

    \item $\mathsf{PubEval}(pk, x, w)$: 以公钥$pk$、实例$x \in L_{pk}$和证据$w$为输入, 输出$y \leftarrow \mathsf{hc}(w)$. 
\end{itemize} 
\end{construction}

\begin{figure}[!hbtp]
\begin{center}
\begin{tikzpicture}
\node [name=domain, circlenode, draw=gray!50, fill=gray!50, minimum size=3em] {$S$}; 
\node [name=W, textnode, below of=domain, yshift=-2.5em] {$W$}; 
\node [name=range, circlenode, draw=blue!20, fill=blue!20, minimum size=3em, right of = domain, xshift=16em] {$U$};
\node [name=hatX, textnode, below of=range, yshift=-2.5em] {$X$}; 
\node [name=Z, circlenode, draw=magenta, fill=magenta, minimum size=3em, above of = domain, yshift=5em] {$K$}; 
\node [name=W, textnode, left of=Z, xshift=-2.5em] {$Y$};  

\node [name=x1, circlenode, draw=black, fill=black, right of = domain, minimum size=0.1em, inner sep = 0, 
    xshift=0.8em, yshift=0.8em] {};
\node [name=y1, circlenode, draw=black, fill=black, left of = range, minimum size=0.1em, inner sep = 0, 
    xshift=-0.8em, yshift=0.8em] {};
\node [name=x2, circlenode, draw=black, fill=black, right of = domain, minimum size=0.1em, inner sep = 0, 
    xshift=0.8em, yshift=-0.8em] {};
\node [name=y2, circlenode, draw=black, fill=black, left of = range, minimum size=0.1em, inner sep = 0, 
    xshift=-0.8em, yshift=-0.8em] {};

\draw (domain) edge[->] node[auto] {$G_{ek}$} (range);
\draw (x1) edge[->, color=blue, bend left=30] node[auto] {$\mathsf{Eval}(ek, x)$} (y1);
\draw (y2) edge[->, color=red, bend left=30] node[below, name=TdInv] {$\mathsf{TdInv}(td, y)$} (x2);
\draw (domain) edge[->] node[auto, name=hc] {$\mathsf{hc}$} (Z);  
\node [name=Gen, textnode, right of = domain, xshift=8em, yshift=7em] 
    {$pp \leftarrow \mathsf{Setup}(1^\kappa)$\\$(ek, td) \leftarrow \mathsf{KeyGen}(pp)$}; 

\node [textnode, name=PubEval, right of = hc, xshift=-6em, color=blue] {$\mathsf{PubEval}$}; 
\draw (PubEval) edge[->] (hc); 

\node [textnode, name=PrivEval, left of = TdInv, xshift=-14.5em, color=red] {$\mathsf{PrivEval}$}; 
\draw (PrivEval) edge[->] (hc); 
\draw (PrivEval) edge[->] (TdInv); 

\end{tikzpicture}
\end{center}
\caption{基于TDF的PEPRF构造}\label{figure:TDF-based-PEPRF}
\end{figure}

构造~\ref{construction:TDF-based-PEPRF}的正确性由陷门单向函数的正确性和单射性保证, 安全性由如下定理保证. 
\begin{theorem}
    如果起点TDF是(自适应)单向的, 那么构造~\ref{construction:TDF-based-PEPRF}中的PEPRF是(自适应)弱伪随机的. 
\end{theorem}


\subsubsection{基于HPS的PEPRF构造}
本章节展示如何基于哈希证明系统构造具有不同安全性质的可公开求值伪随机函数. 

首先展示如何基于平滑HPS构造弱伪随机的PEPRF.  
\begin{construction}[基于平滑HPS的PEPRF构造]\label{construction:SHPS-to-PEPRF}
\begin{itemize}
\item $\mathsf{Setup}(1^\kappa)$: 运行$\text{HPS}.\mathsf{Setup}(1^\lambda)$
    生成HPS的公开参数$pp = (\mathsf{H}, PK, SK, X, L, W, \Pi, \alpha)$, 
    输入PEPRF的公开参数$pp = (F, PK, SK, X, L, W, Y)$, 其中$F = \mathsf{H}$, $Y = \Pi$.  

\item $\mathsf{KeyGen}(pp)$: 运行$(pk, sk) \leftarrow \text{HPS}.\mathsf{KeyGen}(pp)$生成密钥对.  

\item $\mathsf{PrivEval}(sk, x)$: 以私钥$sk$和元素$x \in X$为输入, 计算$y \leftarrow \text{HPS}.\mathsf{PrivEval}(sk, x)$. 

\item $\mathsf{PubEval}(pk, x, w)$: 以公钥$pk$、语言中的元素$x \in L$和相应的证据$w \in W$为输入, 
    计算$y \leftarrow \text{HPS}.\mathsf{PubEval}(pk, x, w)$. 
\end{itemize}
\end{construction}  

\begin{center}
\begin{tikzpicture}
\node [name = domain, circlenode, draw=none, fill=gray!50, minimum size = 5em] {}; 
\node [textnode, above of = domain, yshift=1em] {$X$}; 
\node [name = language, circlenode, draw=none, below of = domain, xshift=0.5em, yshift=-1em, fill = violet!70, minimum size = 2em, inner sep = 0em] {$L$}; 
\node [name = witness, circlenode, below of = language, yshift=-5em, draw=none, fill = cyan, minimum size = 2em, inner sep=0em] {$W$}; 
\node [name = sample, textnode, below of = language, xshift=-5em, yshift = -2.5em] {$\mathsf{SampRel}(r)$};
\draw ($(sample.east)+(0em, 0em)$) edge[->] (language); 
\draw ($(sample.east)+(0em, 0em)$) edge[->] (witness);
\draw (language) edge[<->] node [right] {$\mathsf{R}_L$} (witness);  
\node [name = proof, circlenode, draw=none, fill=blue!20, minimum size = 3em, right of = domain, xshift=14em] {$\Pi$}; 
\node [below of = proof, yshift=-1em] {$Y$}; 
\draw (domain) edge[->] node [above] {$\mathsf{H}_{sk}(x)$} (proof);
\node [name = Gen, right of = domain, xshift=-7em, yshift=4em] 
    {$pp \leftarrow \mathsf{Setup}(1^\kappa)$\\$(pk, sk) \leftarrow \mathsf{KeyGen}(pp)$}; 

\node [name=x1, circlenode, draw=black, fill=black, right of = domain, minimum size=0.1em, inner sep = 0, 
    xshift=1em, yshift=1em] {};
\node [name=y1, circlenode, draw=black, fill=black, left of = proof, minimum size=0.1em, inner sep = 0, 
    xshift=-0.8em, yshift=0.6em] {};
\node [name=x2, circlenode, draw=black, fill=black, right of = domain, minimum size=0.1em, inner sep = 0, 
    xshift=1.2em, yshift=-1.5em] {};
\node [name=y2, circlenode, draw=black, fill=black, left of = proof, minimum size=0.1em, inner sep = 0, 
    xshift=-0.8em, yshift=-0.6em] {};
\draw (x1) edge[color=red, ->, bend left=30] node [above] {$\mathsf{PrivEval}(sk, x)$} (y1);
\draw (x2) edge[color=blue, ->, bend right=30] node [below] {$\mathsf{PubEval}(pk, x, w)$} (y2); 
\end{tikzpicture}
\end{center}
构造~\ref{construction:SHPS-to-PEPRF}的正确性由平滑HPS的正确性保证, 安全性由如下定理保证:
\begin{theorem}
基于$L \subset X$上的SMP假设, 构造~\ref{construction:SHPS-to-PEPRF}中的PEPRF满足弱伪随机性. 
\end{theorem}

PEPRF与HPS在语法上非常相似, 但存在以下微妙的不同, 如表~\ref{table:PRF-vs-PEPRF}所示:
\begin{table}
\centering
\begin{tabular}{|c|c|c|}
\hline
               & HPS & PEPRF \\
\hline
投射性     & $\checkmark$  &  not necessary   \\
\hline
$L$与$X$的关系        & $L \subset X$ & $L \subseteq X$ \\
\hline
弱伪随机性 & $x \sample X \backslash L$  &  $x \sample L$     \\
\hline
\end{tabular}
\caption{HPS与PEPRF的不同}\label{table:HPS-vs-PEPRF}
\end{table}

下面展示如何基于平滑和一致HPS构造自适应伪随机的PEPRF. 
\begin{construction}[基于平滑和一致HPS的PEPRF构造]\label{construction:SUHPS-to-PEPRF}
\begin{trivlist}
\item 构造组件: 针对同一语言$\tilde{L} \subset \tilde{X}$的smooth $\text{HPS}_1$和2-universal $\text{HPS}_2$:
\end{trivlist}
构造如下:  
\begin{itemize}
\item $\mathsf{Setup}(1^\kappa)$: 运行$pp_1 = (\mathsf{H}_1, PK_1, SK_1, \tilde{X}, \tilde{L}, W, \Pi_1, \alpha_1) \leftarrow 
    \text{HPS}_1.\mathsf{Setup}(1^\kappa)$生成smooth HPS的公开参数, 
    运行$pp_2 = (\mathsf{H}_2, PK_2, SK_2, \tilde{X}, \tilde{L}, \tilde{W}, \Pi_2, \alpha_2) \leftarrow 
    \text{HPS}_2.\mathsf{Setup}(1^\kappa)$生成2-universal HPS的公开参数, 
    基于$pp_1$和$pp_2$生成PEPRF的公开参数$pp = (F, PK, SK, X, L, W, Y)$, 
    其中$X = \tilde{X} \times \Pi_2$, $Y = \Pi_1 \cup \bot$, $PK = PK_1 \times PK_2$, 
    $L = \{L_{pk}\}_{pk \in PK}$定义在$X = \tilde{X} \times \Pi_2$上, 
    其中$L_{pk} = \{x = (\tilde{x}, \pi_2): \exists w \in W \text{~s.t.~}
    \tilde{x} \in \tilde{L} \wedge \pi_2 = \text{HPS}_2.\mathsf{PubEval}(pk_2, \tilde{x}, w)\}$, 
    相应的采样算法$\mathsf{SampRel}$以公钥$pk = (pk_1, pk_2)$和随机数$r$为输入, 首先随机采样语言$\tilde{L}$的随机实例证据元组$(\tilde{x}, w)$, 
    计算$\pi_2 \leftarrow \text{HPS}_2.\mathsf{PubEval}(pk_2, \tilde{x}, \tilde{w})$, 
    输出语言$L$的实例$x = (\tilde{x}, \pi_2)$和证据$w = \tilde{w}$.  
    不失一般性, 令$pp$包含$pp_1$和$pp_2$中的所有信息. 

\item $\mathsf{KeyGen}(pp)$: 从$pp$中解析出$pp_1$和$pp_2$, 
    运行$(pk_1, sk_1) \leftarrow \text{HPS}_1.\mathsf{KeyGen}(pp_1)$和$(pk_2, sk_) \leftarrow \text{HPS}_2.\mathsf{KeyGen}(pp_2)$,  
    输出$pk = (pk_1, pk_2)$和$sk = (sk_1, sk_2)$.  

\item $\mathsf{PrivEval}(sk, x)$: 以私钥$sk = (sk_1, sk_2)$和$x = (\tilde{x}, \pi_2)$为输入, 
    如果$\pi_2 = \text{HPS}_2.\mathsf{PrivEval}(sk_2, \tilde{x})$则返回$\bot$ 
    否则返回$y \leftarrow \text{HPS}_1.\mathsf{PrivEval}(sk_1, \tilde{x})$. 
    该算法定义了$F: SK \times X \rightarrow Y \cup \bot$.  

\item $\mathsf{PubEval}(pk, x, w)$: 以公钥$pk = (pk_1, pk_2)$、元素$x = (\tilde{x}, \pi_2) \in L_{pk}$以及证据$w$为输入, 
    输出$y \leftarrow \text{HPS}_1.\mathsf{PubEval}(pk_1, \tilde{x}, w)$. 
\end{itemize}  
\end{construction}

\begin{theorem}\label{theorem:HPS-to-APEPRF}
    基于$\tilde{L} \subset \tilde{X}$上的SMP假设, 构造~\ref{construction:SUHPS-to-PEPRF}中的PEPRF是自适应弱伪随机的.
\end{theorem}


\begin{remark}
构造~\ref{construction:SHPS-to-PEPRF}相对直接, 构造~\ref{construction:SUHPS-to-PEPRF}稍显复杂, 
其中蕴含的设计思想与基于哈希证明系统构造CCA-secure PKE相似: 使用``弱''HPS封装随机会话密钥, 
使用``强''HPS生成证明以杜绝``危险''解密询问.       
\end{remark}

\subsubsection{基于EHPS的PEPRF构造}
本章节展示如何基于(ABO-)EHPS构造PEPRF.
\begin{construction}[基于(ABO-)EHPS的PEPRF构造]\label{construction:EHPS-based-PEPRF}
\begin{itemize}
\item $\mathsf{Setup}(1^\kappa)$: 运行$pp = (\mathsf{H}, PK, SK, \tilde{L}, \tilde{W}, \Pi) 
    \leftarrow \text{EHPS}.\mathsf{Setup}(1^\kappa)$生成EHPS的公开参数, 
    令$\mathsf{hc}: \tilde{W} \rightarrow Z$是单向关系$\mathsf{R}_{\tilde{L}}$的hardcore function; 
    生成PEPRF的公开参数$pp = (F, PK, SK, X, L, W, Y)$, 其中$X = \tilde{L} \times \Pi$, $Y = Z$, $W = R$, 
    $L = \{L_{pk}\}_{pk \in PK}$定义在$X = \tilde{L} \times \Pi$上, 
    其中$L_{pk} = \{x = (\tilde{x}, \pi): \exists w \in W \text{~s.t.~} \tilde{x} = \mathsf{SampIns}(w) 
    \wedge \pi = \text{EHPS}.\mathsf{PubEval}(pk, \tilde{x}, w)\}$, 
    相应的采样算法以公钥$pk$和随机数$w$为输入, 首先以$w$作为证据生成实例$\tilde{x} \sample \tilde{L}$, 
    再计算$\pi \leftarrow \text{EHPS}.\mathsf{PubEval}(pk, \tilde{x}, w)$, 输出实例$x = (\tilde{x}, \pi)$和证据$w$. 
    $F_{sk}(x):= \mathsf{hc}(\text{EHPS}.\mathsf{Ext}(sk, x))$.   

\item $\mathsf{KeyGen}(pp)$: 运行$(pk, sk) \leftarrow \text{EHPS}.\mathsf{KeyGen}(pp)$生成密钥对. 

\item $\mathsf{PrivEval}(sk, x)$: 以私钥$sk$和$x \in X$为输入, 将$x$解析为$(\tilde{x}, \pi)$, 
    计算$\tilde{w} \leftarrow \text{EHPS}.\mathsf{Ext}(sk, \tilde{x}, \pi)$, 
    输出$y \leftarrow \mathsf{hc}(\tilde{w})$.   

\item $\mathsf{PubEval}(pk, x, w)$: 以公钥$pk$、$x \in L_{pk}$以及相应的证据$w$为输入,  
    计算$\tilde{w} \leftarrow \mathsf{SampWit}(w)$, 输出$y \leftarrow \mathsf{hc}(\tilde{w})$.   
\end{itemize}
\end{construction} 

\begin{theorem}
    如果$\mathsf{R}_{\tilde{L}}$是单向的, 基于(ABO-)EHPS的PEPRF是(自适应)弱伪随机的.
\end{theorem}

\begin{figure}[!hbtp]
\begin{center}
\begin{tikzpicture}
\node [name = Setup, textnode] {$pp \leftarrow \mathsf{Setup}(1^\kappa)$\\$(pk, sk) \leftarrow \mathsf{KeyGen}(pp)$}; 

\node [name = witness, circlenode, below of = Setup, xshift=-8em, yshift=-4em, draw = none, fill = cyan, minimum size = 3em] {$\tilde{W}$};
\node [name = language, circlenode, draw=none, fill=violet!70, minimum size = 3em, below of = witness, yshift=-12em] {$\tilde{L}$}; 
\node [name = range, circlenode, left of = witness, xshift = -8em, 
    draw = none, fill = magenta, minimum size = 3em] {$Z$};

\node [name = proof, circlenode, right of = language, xshift=16em, draw = none, fill = blue!20, minimum size = 3em] {$\Pi$};

\node [name = sample, textnode, above of = language, yshift=6em, xshift=-12em] {$\mathsf{SampRel}(w)$}; 


\draw ($(sample.east)+(0em, 0em)$) edge[->] node[fill=white] {$\mathsf{SampWit}(w) \rightarrow \tilde{x}$} (language); 
\draw ($(sample.east)+(0em, 0em)$) edge[->] node[fill=white] {$\mathsf{SampIns}(w) \rightarrow \tilde{w}$} (witness);
\draw (language) edge[<->] node [right] {$\mathsf{R}_L$} (witness); 
\draw (witness) edge[->] node [above] {$\mathsf{hc}$} (range); 

\draw (language) edge[->] node [above] {$\mathsf{H}_{pk}(\tilde{x})$} (proof);
\draw (language) edge[->, bend left=35, color=blue] node[above] {$\mathsf{PubEval}(pk, \tilde{x}, w)$} (proof);  
%\draw (language) edge[->, bend left=35, color=red] node[below] {$\mathsf{PrivEval}(sk, x)$} (proof); 

\draw (proof) edge[->, bend right=30, color=blue] node[above, xshift=1em, yshift=1em] {$\mathsf{Ext}(sk, \tilde{x}, \pi)$} (witness); 

\draw (language) edge[->, bend right=50, color=blue] (witness);   
\end{tikzpicture}
\end{center}
\caption{基于EHPS的PEPRF构造}\label{figure:EHPS-based-PEPRF}
\end{figure}

\subsubsection{基于$\iO$的PEPRF构造}
本章节展示如何基于不可区分程序混淆构造PEPRF.
\begin{construction}[基于$\iO$和PPRF的PEPRF构造]\label{construction:iO-based-PEPRF}
\begin{trivlist}
    \item 构造组件: 不可区分程序混淆$\iO$、伪随机数发生器和可穿孔伪随机函数
\end{trivlist}
构造如下: 
\begin{itemize}
\item $\mathsf{Setup}(1^\kappa)$: 生成对电路族$\mathcal{C}_\kappa$的不可区分程序混淆$\iO$, 
        选取长度倍增的伪随机数发生器$\mathsf{PRG}: \{0, 1\}^{\kappa} \rightarrow \{0,1\}^{2\kappa}$和可穿孔伪随机函数
        $\mathsf{PPRF}: K \times \{0,1\}^{2\kappa} \rightarrow Y$; 
        生成PEPRF的公开参数$pp = (F, PK, SK, X, L, W, Y)$,  
        其中$PK = \iO(\mathcal{C}_\kappa)$, $SK = K$, $X = \{0,1\}^{2\kappa}$, 其中$F:=\mathsf{PPRF}$,  
        $W = \{0,1\}^\kappa$, $L = \{x \in X: \exists w \in W \text{~s.t.~} x= \mathsf{PRG}(w)\}$, 
        相应的采样算法$\mathsf{SampRel}$以随机数$r \in \{0,1\}^\kappa$为输入, 
        输出实例$x \leftarrow \mathsf{PRG}(r)$和证据$w = r$.   

    \item $\mathsf{KeyGen}(pp)$: 随机采样$k \in K$作为私钥$sk$, 计算$pk \leftarrow \iO(\mathsf{Eval})$作为公钥. 

    \item $\mathsf{PrivEval}(sk, x)$: 输出$y \leftarrow \mathsf{PPRF}(sk, x)$. 

    \item $\mathsf{PubEval}(pk, x, w)$: 将公钥$pk$解析为程序, 计算$y \leftarrow pk(x, w)$.   
\end{itemize}
\end{construction}   

\begin{figure}[!hbtp]
\centering
\begin{tikzpicture}
\node [name = domain, circlenode, minimum size = 6em, draw=none, fill=gray!50] {}; 
\node [textnode, above of = domain, yshift=1.5em] {$X$}; 
\node [textnode, above of = domain, yshift=4em] {$\{0,1\}^{2\lambda}$}; 
\node [name = language, circlenode, below of = domain, yshift=-1em, minimum size = 3em, draw=none, fill=violet!70] {$L$}; 
\node [name = witness, circlenode, below of = language, yshift=-8em, minimum size = 3em, draw=none, fill=cyan] {$W$}; 
\node [name = range, circlenode, right of = domain, xshift=16em, minimum size = 3em, draw=none, fill=blue!20] {$Y$}; 
\node [textnode, below of = witness, yshift=-2em] {$\{0,1\}^\lambda$}; 
\draw (language) edge[<->] node[left] {$\mathsf{PRG}$} (witness); 

\node [name = x1, circlenode, right of = domain, draw=black, fill=black, 
    xshift=1em, yshift=1em, minimum size = 0.1em, inner sep=0em] {};  
\node [name = y1, circlenode, left of = range, draw=black, fill=black, 
    xshift=-0.5em, yshift=0.6em, minimum size = 0.1em, inner sep=0em] {}; 

\node [name = x2, circlenode, right of = domain, draw=black, fill=black, 
    xshift=0.8em, yshift=-1em, minimum size = 0.1em, inner sep=0em] {};  
\node [name = y2, circlenode, left of = range, draw=black, fill=black, 
    xshift=-0.5em, yshift=-0.6em, minimum size = 0.1em, inner sep=0em] {};  

\draw (domain) edge[->] node[above] {$\mathsf{PPRF}_{sk}(x)$} (range);
\draw (x1) edge[->, bend left=35, color=red] node[above, color=red] {$\mathsf{PrivEval}(sk, x)$} (y1); 
\draw (x2) edge[->, bend right=35, color=blue] node[below, color=blue, name=PubEval] {$\mathsf{PubEval}(pk, x, w)$} (y2);

\node [name=pk-program, rectanglenode, right of=PubEval, xshift=16em, yshift=-2em, minimum width=18em, minimum height=8em] {
\begin{minipage}{16em}
\footnotesize{
\begin{center}
    $\mathsf{Eval}$
\end{center}
\begin{trivlist}
    \item Constants: PPRF key $sk$. 
    \item Input: $x \in \{0,1\}^{2\lambda}$, $w \in \{0,1\}^\lambda$. 
        \begin{enumerate}\itemsep 1pt \parskip 0pt \parsep 0pt
            %\item output $\mathsf{PPRF}(sk, \mathsf{PRG}(w))$. 
            \item output $\mathsf{PPRF}(sk, x)$ if $x = \mathsf{PRG}(w)$. 
            \item else, output $\bot$. 
        \end{enumerate}
\end{trivlist}
}
\end{minipage}
}; 
\draw ($(PubEval.south)+(0.5em, 0em)$) edge[->, bend right=30, dashed] (pk-program.west);  
\end{tikzpicture}
\end{figure}

\begin{theorem}
基于不可区分程序混淆、伪随机数发生器和可穿孔伪随机函数的安全性, 构造~\ref{construction:iO-based-PEPRF}中的PEPRF满足自适应弱伪随机性. 
\end{theorem}

\begin{remark}
    上述构造实质上展示了$\iO$可以将$\mathsf{Minicrypt}$中的可穿孔伪随机函数编译为$\mathsf{Cryptomania}$中的可公开求值伪随机函数.
\end{remark}


\subsection{小结}
本章中引入了PEPRF这一全新的密码组件, 展示了它与已有密码组件之间的联系以及它的应用, 如图~\ref{figure:web-of-PEPRF}所示. 
引入PEPRF最大的理论意义在于它不仅首次阐明了经典的Goldwasser-Micali PKE和ElGamal PKE的构造机理, 
还统一了几乎所有已知的构造范式. 
作为首个实用的公钥加密, RSA PKE影响深远, 令单向陷门函数的概念深入人心, 使得人们常有``构造公钥加密必须有陷门''的错觉. 
PEPRF树立了正确的认知, 指出构造公钥加密的实质在于构造可公开求值的伪随机函数, 核心技术是``令同一函数存在两种求值方法''.  
基于PEPRF的PKE构造恰与Minicrypt中基于PRF的SKE构造形成完美的形式契合与思想共鸣. 

\begin{figure}[!hbtp]
\begin{center}
\begin{tikzpicture}
    \node [name=PEPRF, draw=none, roundnode, fill=magenta!20, minimum width=5em, minimum height=2em] {PEPRF};

    \node [name=TDF, textnode, above of=PEPRF, xshift=-12em, yshift=8em] {TDF}; 
    \node [name=EHPS, textnode, above of=PEPRF, xshift=-4em, yshift=8em] {EHPS};
    \node [name=HPS, textnode, above of=PEPRF, xshift=4em, yshift=8em] {HPS};
    \node [name=iOPPRF, textnode, above of=PEPRF, xshift=10em, yshift=8em] {$\iO$+PRG+PPRF}; 

    \node [name=PSPRF, draw=none, roundnode, below of = PEPRF, yshift=-6em, fill=orange!20, minimum width=5em, minimum height=2em] {PSPRF};
    \node [name=KEM, textnode, below of=PSPRF, yshift=-4em] {PKE/KEM};
    \draw (PEPRF) edge[->, thick] (PSPRF);
    \draw (PSPRF) edge[->, thick] (KEM);

    \draw (EHPS) edge[->, thick] (PEPRF);
    \draw (iOPPRF) edge[->, thick] (PEPRF);
    \draw (HPS) edge[->, thick] (PEPRF);
    \draw (TDF) edge[->, thick] (PEPRF);

    \draw[decorate, decoration={brace, mirror, raise=2pt, amplitude=6pt}, thin]
    ($(PEPRF.east)+(1.5em, 2em)$)--($(PEPRF.east)+(1.5em, -2em)$); 

    \node [textnode, name=DDH, right of = PEPRF, xshift=9em, yshift=2em] {DDH $\leadsto$ ElGamal PKE}; 
    \node [textnode, name=QR, right of = PEPRF, xshift=10.5em, yshift=-2em] {QR $\leadsto$ Goldwasser-Micali PKE}; 

    \node [name=TDR, textnode, left of=PEPRF, xshift=-8em] {TDR};
    \draw (TDF) edge[->, thick] (TDR);
    \draw (EHPS) edge[->, thick] (TDR);

    \draw (TDR) edge[->, thick] (PSPRF);

    \node [name=PRF, rectanglenode, fill=blue!20, minimum width=5em, minimum height=2em, right of = PEPRF, xshift=25em] {PRF};
    \node [name=SKE, textnode, below of=PRF, yshift=-10em] {SKE};
    \draw (PRF) edge[->, thick] (SKE);  

    \node [name=OWF, textnode, above of = PRF, yshift=8em] {OWF};
    \draw (OWF) edge[->, thick] (PRF); 

    \node [textnode, name=cryptomania, above of = PEPRF, yshift=12em, color=magenta] {Cryptomania};
    \node [textnode, name=minicrypt, above of = PRF, yshift=12em, color=blue] {Minicrypto}; 

    \draw ($(cryptomania.east)+(15em, 1em)$) edge[dotted] ($(cryptomania.east)+(15em, -23em)$); 


\end{tikzpicture}
\end{center}
\caption{PEPRF的构造与应用}\label{figure:web-of-PEPRF}
\end{figure}

PEPRF的强大威力来源于其高度抽象, 它诠释了公钥加密设计的``万法同源, 殊途同归''. 

\begin{note}
抽象的概念是美妙的, 相信读者能够通过PEPRF感受到``大繁至简''的优雅与``高屋建瓴''的力量. 
然而抽象概念是果, 具体构造是因. 切不能刻意过度的抽象而忽视具体构造, 正是多种多样具体构造才让我们能够有机缘洞见事物本质, 使得高度凝练的概念内涵丰富、意义深刻. 
\end{note}

