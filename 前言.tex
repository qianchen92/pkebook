\chapter{前言}
本书是在``2017-2019中国科学院大学研究生暑期课程''和``2022北京大学应用数学专题讲习班''的讲义基础上, 结合多年在公钥加密方面的研究成果编写而成. 
目的是尽快引导读者到达公钥加密这一极重要的现代密码学领域, 力求从高屋建瓴的视角介绍公钥加密的设计方法, 对重要的思想剥丝抽茧、对关键的技术条分缕析. 
写作过程中根据教学和研究经历感悟对内容做了取舍和精简, 参阅了国际顶级会议期刊的前沿论文, 也融入了作者的独立思考.   

计算机网络技术的飞速发展引发了人类社会组织形态的根本性变革, 从集中式迁移为分布式, ``海内存知己, 天涯若比邻''从诗歌意象走进现实世界. 
面向分布式环境下的隐私保护需求, 1976年Diffie和Hellman开创了现代密码学的新方向—公钥密码学. 
迄今为止的半个多世纪以来, 公钥密码学一直处于最活跃的前沿, 引领驱动了密码学的研究进展, 极大丰富了密码学的学科内涵. 
公钥加密作为公钥密码学最重要的分支, 在理论方面孕育了可证明安全方法、将各类数学困难问题纳入工具库、启发了一系列密码原语和重要概念, 
已有多项历史性成果获得Turing奖和G\"odel奖; 在应用方面则是各类网络通信安全协议的核心组件, 时刻保护着公开信道上消息传输的机密性. 

近期, 公钥加密仍处于快速发展阶段, 在安全性方面, 各类超越传统语义安全的高级安全属性研究已经日趋成熟, 基于复杂性弱假设的细粒度安全的研究正在兴起; 
在功能性方面, 函数加密的研究方兴未艾, 全同态加密的研究如火如荼. 我们已经有幸见证了公钥加密之旅的美妙风景, 但还有更广袤深邃的领域待探索征服. 

公钥加密历经多年发展, 各类方案层出不穷, 相关概念定义繁多, 因此想深入学习的读者往往会感觉陷入书山文海, 难识庐山真面目. 
本书试图快速引导读者登高俯瞰, 将公钥加密的设计方法尽收眼底, 达到万变不离其宗的认知. 
为此, 本书的内容偏重基于一般假设的通用构造. 这样的选择有诸多好处, 从理论角度, 通用构造剥离了不必要的细节、凸显了核心要素, 从而更容易洞察公钥加密的本质和复杂性下界; 
从实际角度, 通用构造可以启发更多的具体构造, 可根据安全或应用的需求灵活选择新的困难假设给出实例化方案. 

本书的第一章简述了公钥加密的发展历程, 第二章介绍了准备知识, 为后续章节做好铺垫. 
第三章回顾经典的公钥加密方案, 方便读者先获得具象的认识, 预备一些重要的例子. 
第四章是核心部分, 从各类密码组件出发展示了公钥加密的通用构造, 并在最后获得更高阶的抽象, 与对称加密的构造相互呼应、完美契合. 
第五章和第六章分别从安全性增强和功能性扩展两个维度介绍公钥加密的重要成果和前沿进展. 
第七章简介了公钥加密的标准化与工程实践, 打通理论与实践的最后一公里. 

书中有不少看似不起眼的注记, 它们大多来源于作者科研过程中的心得体会, 领悟其中蕴含的思辨方式之后或许能看到更美的风景. 
总的来说, 切实掌握本书的内容之后, 可使读者进入可证明安全密码学的新层次, 为进一步的研究打好基础. 

密码科学技术国家实验室对本书的出版给予了极大支持, 作者深表感谢. 
作者也借此机会对在密码学研究给予作者热情支持帮助的上海交通大学的郁昱教授、刘胜利教授和山东大学的王美琴教授深表感谢. 
最后, 感谢家人们的默默支持, 没有你们的全情支持, 我们不可能完成此书.   

由于水平有限, 时间紧迫, 定有许多不当之处. 诚恳欢迎批评指正. \\*[2em]

\hfill 陈宇 \& 秦宝东 \\

\hfill 2023年夏  